\pagebreak
\section{Random Forest}
%Decision trees, which we mentioned in the previous section, have been used for a long time. 
Deployment of Decision Trees is visible in simple situations and also in more complex scientific or real life and industrial affairs.
The recent popularity of decision trees is due to work presented by 
\cite{Breiman1996OUT-OF-BAG-E}, \cite{breiman2001random}, and \cite{breiman2004consistency} that ensembles of 
different decision trees can get a meaningful improvement in accuracy in classification problems and other 
common learning tasks such as regression. As the unit in this procedure is based on Decision Tree and includes an injection of
randomness, this method is known as Random Forest. 

\subsection{Main Idea and Illustration}
An ensemble of randomly trained Decision Trees, i.e. Random Forest, was defined by \cite{breiman2001random}:

\newtheorem{definition}{Definition}
\begin{definition}
	Random Forest is a classifier consisting of a collection of tree-structured classifiers ${\hat{T}_{\theta_{b}}(\textbf{x})}, b = 1,...,B$ where the $\theta_{b}$ are independent identically
	distributed random vectors and each tree casts a unit vote for the most popular class at input $\textbf{x}$ .
\end{definition}

\begin{algorithm}[H]
\SetAlgoLined
\begin{enumerate}
	\item For $b$ = 1 to $B$:
	\begin{enumerate}
	    \item Draw a bootstrap sample $D_{b}$ of size N from the training data.
	    \item Grow the Random Forest tree ${{T}_{D_{b},\theta_{b}}}$ to the bootstrapped data, by recursively repeating the following steps for each terminal node of the tree, until the minimum node size $n_{min}$ is reached:
	    \begin{enumerate}
	       \item Select $m$ variables denoted by $\theta_{b}$ at random from the $n$ variables
	       \item Pick the best variable/split-point among the $m$
	       \item  Split the node into two daughter nodes
	    \end{enumerate}
	\end{enumerate}
	\item  Output the ensemble of trees $\{{T}_{D_{b},\theta_{b}}\}_{b=1}^{B}$
\end{enumerate}
 \caption{Random Forest for Regression or Classification \cite{friedman2001elements}}
\end{algorithm}

\subsubsection{Randomness in the Model}
The main aspect of Random Forest model is an injection of randomness which allows being all unit trees different from the others. Two key concepts make decision forest "random" are:
\begin{enumerate}
	\item Random sampling of training data points when building trees
	\item Random subsets of variables considered when splitting nodes
\end{enumerate}

In practice, random sampling of training observations is done by bootstrapping. 
Bootstrapping for random forests means that every single decision tree trained on a random sample which is drawn with replacement. 
The second important injection of randomness into the model is taking into consideration only a subset of all the variables 
when splitting each node in every Decision Tree. 
Number of variables for splitting the node has to be chosen and 
as mentioned in \cite{friedman2001elements}, following specifications are recommended:
\begin{enumerate}
	\item For classification:  $\lfloor{\sqrt{n}} \rfloor$ and the minimum node size is one
	\item For regression: $\lfloor \frac{n}{3} \rfloor$ and the minimum node size is five
\end{enumerate}
where $n$ is a number of variables in the classification or regression problem. 

To exemplify, in classification problem with 10 different variables, only randomly chosen 3
of them will be considered for splitting the node. 
These aforementioned specifications are
only recommendations \cite{friedman2001elements}\cite{breiman2001random}. In practice, 
the most suitable values for these parameters
can differ and they will depend on the problem. Therefore, these parameters should be
treated as tuning parameters. Randomness parameter, the number of chosen variables,
has a decisive impact on the model since it controls not only the amount of randomness
within each tree but also the amount of correlation between different trees in the forest.
As randomness parameter decreases, trees become more decorrelated  \cite{criminisi2012decision}.

\subsubsection{Training, Testing and Prediction}\label{tra_test_pred}
Training and testing is an integral part of building up every machine learning model.
In Random Forest, training of all trees is done independently and testing consists in the fact that each test point $\textbf{x}$ is pushed 
through every tree included in forest until it ends in corresponding leave. As a last step is taking all predictions from 
every decision tree and combining them into single prediction of Random Forest. 
We can adopt various approachs when combining the Decision Tree predictions. We will delve into details regarding those approachs but in contemporary exercises, Random Forests generate probabilistic output in classification problems \cite{CGV-035}. 
They return not just a single class point prediction, but whole class distribution and the combination of tree predictions can be described as below:
\begin{equation}
	p(c|\textbf{x}) =  \frac{1}{B} \displaystyle\sum_{b=1}^{B} p_{b}(Y = c |X = \textbf{x})
\end{equation}
where in Random Forest with the number of decision trees equals to $B$ each tth tree obtains the posterior 
distribution $ p_{b}(Y = c |X = \textbf{x}) $.

\subsubsection{Out of Bag Sample}
In applications of Random Forest, we can use Out of Bag sample to validate our model.
When performing a bootstrap for getting a sample of data for training, in general only $2/3$ of data is included and the
left-out part of data is called Out of Bag (OOB) sample.
After training the Random Forest model, we use the OOB sample used as a validation set. 
Every Decision Tree in Random Forest is validated with the corresponding leftover sample 
and averaged to calculate Out of Bag estimate \cite{friedman2001elements}. 
Out of Bag sample is commonly used to monitor error of predictions. 
Breiman’s work \cite{Breiman1996OUT-OF-BAG-E} shows empirically that Out of Bag estimate is as accurate 
as a test set of the same size as the training set \cite{Breiman1996OUT-OF-BAG-E}.

\subsection{Mathematical Explanation}
Let $D = \{(x_{1},y_{1}), (x_{2}, y_{2}), ... , (x_{N}, y_{N})\}$ 
be the set we want to train our model on and $T_{D, \theta}$ be 
Decision Tree produced by using the set $D$ and parameters $\theta$. 
We assume that $D$ is countable which normally is the 
case especially for class values, although replacing sums with integrals can extend the analysis and 
provide results for uncountable sets as well \cite{kohavi1996bias}. As mentioned earlier, 
Random Forest classifier selects a bootstrapped subset of observations and 
grow Decision Trees with only a subset of regressors. 
Repeating this tree growing process $B$ times 
yields a Random Forest. Assume $x^*$ is the value that we want to predict its class, 
there are two rules that can be used to get the prediction; majority voting and 
soft voting \cite{louppe2014understanding}\cite{zhou2012ensemble}.

In majority voting, after getting every trees prediction denoted as $\hat{T}(x^*)$ final prediction is the class that gets most votes from trees:
\begin{equation}
\boldsymbol{RF}_{D, \theta_{1}, \theta_{2}, ..., \theta_{B}} (x^*) =
	\underset{c \in Y}{argmax} \sum_{b = 1}^{B}{1(\hat{T}_{b}(x^*) = c)}
\end{equation}
We already explained identification of the posterior distribution of classes in section \ref{tra_test_pred}. 
After estimating the probability estimates of a Decision Tree $b$ 
denoted as $\hat{p}_{D, \theta_{b}} (Y = c | X = x^*)$ and 
with averaging those probability estimates, the most likely class is chosen as prediction 
namely;
\begin{equation}
\boldsymbol{RF}_{D, \theta_{1}, \theta_{2}, ..., \theta_{B}} (x^*) =
	\underset{c \in Y}{argmax} \dfrac{1}{B}\sum_{b = 1}^{B}{\hat{p}_{D, \theta_{b}} (Y = c | X = x^*)}
\end{equation}
As mentioned in \cite{breiman1996bagging}, the two aforementioned voting procedures provides similar results, 
yet using soft voting can provide smoother class probability estimates and be exploited in a deeper analysis setting such as 
certainty estimates investigation \cite{louppe2014understanding}. 
We introduced majority voting to enhance our understanding and will use soft voting in further derivations. 

\subsubsection{Properties}
Having explained the details of the algorithm, we can explicitly show 
the improvement of Random Forest upon decision trees 
with exploiting the generalization error. 
We can measure of a model's fit with the generalization error which also called test error or the expected prediction error, 
we will start with examining a decision tree's generalization error and expand our finding to Random Forest. 
The derivations and findings in this section closely follow Louppe's paper \cite{louppe2014understanding}.
The generalization error of a decision tree $T_{D,\theta}$ which is grown using the set $D$ and parameters $\theta$ is;
\begin{equation}
	\boldsymbol{Err}(T_{D,\theta}) = \mathbb{E}_{X,Y}\{L(Y, T_{D,\theta}(X)) \}
\end{equation}
where $L$ is the loss function measuring the difference between its two arguments. 
Since we focus on classification setting, 
the zero-one loss function is our interest, however, 
to get a better understanding, widely used in regression type predictions, 
the squared loss function will be examined jointly. 
The bias-variance decomposition of both functions are similar and 
follow the same dynamics \cite{domingos2000decomposition}. 
While the bias-variance decomposition of zero-one loss function remains 
to be relatively unexplanatory, the squared loss provides us a superior insight. 
Therefore we denoted the findings of the zero-one loss function with an apostrophe.
The squared loss function measures the squared difference between the dependent variable and its predicted value 
by decision tree $T_{D,\theta}$ and can be defined as
\begin{equation}
	L(Y, T_{D, \theta}(x)) = (Y - T_{D, \theta}(x))^2
\end{equation}
while the zero-one loss function yielding
\begin{equation}
	L'(Y, T_{D,\theta}(X)) = \mathds{1} (Y \neq T_{D, \theta}(X))
\end{equation}
$L'(Y, T_{D,\theta}(X))$ equals to $1$ if the class of dependant variable is 
different than its predicted class by decision tree and equals to 0 if both are the same, 
meaning that $Y = T_{D, \theta}(X)$. 
Both loss functions follow the same logic in essence. 
While the squared loss function yielding a number to measure the error, 
the zero-one loss function only yields $1$ or $0$.
The generalization error for both functions 
\begin{align}
	\boldsymbol{Err}(T_{D,\theta}) & = \mathbb{E}_{X,Y}\{ (Y - T_{D, \theta}(x))^2 \} \\
	\boldsymbol{Err'}(T_{D,\theta}) & = \mathbb{E}_{X,Y}\{ 1(Y \neq T_{D, \theta}(X)) \}
	= P(Y \neq T_{D, \theta}(X))
\end{align}
where the expression in the first equation measures the squared difference of true and predicted values
and the expression in the second equation is the probability of misclassification of the tree. 
We will decompose the generalization error and investigate the improvement of Random Forest upon 
Decision Tree.
\vspace{2mm}
\\
\textbf{\emph{The Decomposition of $\boldsymbol{Err}(T_{D,\theta})$}}\\
Given the probability distribution of P(X,Y), 
there exists a model $\phi_{\beta}$ that minimizes the generalization error 
and it can be derived analytically \cite{louppe2014understanding}. 
With conditioning on X generalization error for $\phi_{\beta}$ becomes;
\begin{equation}
\mathbb{E}_{X,Y} \{L(Y, \phi_{\beta}(X))\} = \mathbb{E}_{X}\{\mathbb{E}_{Y|X}\{L(Y, \phi_{\beta}(X)) \} \}
\end{equation}
Point-wise minimization inner term  with respect to Y yields;
\begin{equation}
\phi_{\beta} = \underset{c \in Y}{argmin} \; \mathbb{E}_{Y|X=x}\{L(Y,c)\}
\end{equation}
$\phi_{\beta}$ is defined as Bayes Model and $\boldsymbol{Err}(\phi_{\beta})$ is the residual error, 
the minimum obtainable error with any model. 
The irreducible error, $\boldsymbol{Err}(\phi_{\beta})$, arises solely 
due to random deviations in the data \cite{louppe2014understanding}. 
We will exploit the irreducible concept when examining the dynamics of Random Forest and 
Decision Tree. In that sense, with couple of manipulations, Bayes Model for squared loss is
\begin{align}
\phi_{\beta} & = \underset{c \,\in\, Y}{argmin} \; \mathbb{E}_{Y|X=x}\{(Y-c)^2 \} \notag\\
			 & = \mathbb{E}_{Y|X=x}\{Y\}
\end{align}
For squared loss function, Bayes Model predicts the expected value of Y at X=x 
since the mean of $Y$ minimizes the squared difference 
and equals to its expected value.
Bayes Model of zero-one loss function denoted as $\phi_{\beta}'$ is
\begin{align}\label{eq:bayes_model}
\phi_{\beta}' & = \underset{c \,\in\, Y}{argmin} \; \mathbb{E}_{Y|X=x}\{L(Y,c) \} \notag\\
			 & = \underset{c \,\in\, Y}{argmax} \; P(Y = c \,| X = x)
\end{align}
The class with highest probability in the set $Y$ is chosen by Bayes Model when using the zero-one loss function, 
we included intermediate steps in section \ref{app:bayes_model}. 
Aforementioned residual error can be computed for both functions:
\begin{align}
\boldsymbol{Err}(\phi_{\beta}) & = \mathbb{E}_{Y|X=x}\{(Y-\phi_{\beta}(x))^2 \}\\
\boldsymbol{Err}(\phi_{\beta}') & = P(Y \neq \phi_{\beta}'(x) )
\end{align}
With using the squared loss function, $\boldsymbol{Err}(T_{D,\theta}(x))$ can be written as
\begin{align}\label{eq:decomp_squared_loss}
\boldsymbol{Err}(T_{D, \theta}(x)) & = \mathbb{E}_{Y|X=x}\{(Y-T_{D,\theta}(x))^2\} \notag \\
							   	  & = \boldsymbol{Err}(\phi_{\beta}(x)) + (\phi_{\beta}(x)-T_{D,\theta}(x))^2
\end{align}
As mentioned above, first term in the equation corresponds the 
irreducible error and the second term is due to the prediction differences 
between Bayes Model and our decision tree estimation. 
The generalization error increases with an increase in that difference. 
Since the result does not depend on the Y-values, 
it can be also expressed without conditional expectation. 
Decision Trees in Random Forest classifier uses a bootstrapped dataset 
$D$, thus, if we further examine the second term with taking expectation over $D$, it decomposes as
\begin{align}\label{eq:decomp_squared_loss_cont}
	& \mathbb{E}_{D}\{(\phi_{\beta}(x) - T_{D,\theta}(x))^2 \} \notag \\
	& \: = [\phi_{\beta}(x) - \mathbb{E}_{D}\{T_{D,\theta}(x)\}]^2 + 
	\mathbb{E}_{D}\{[\mathbb{E}_{D}\{T_{D,\theta}(x)\} - T_{D,\theta}(x)]^2\}
\end{align}
With intermediate steps being in section \ref{app:bias_var_decomp}, 
the first term in equation (\ref{eq:decomp_squared_loss_cont}) 
also called squared bias ($bias^2$) shows how the expected prediction of our 
decision tree differs from Bayes Model and the latter term is the variance of our estimator. 
Therefore, we can define $\boldsymbol{Err}(T_{D,\theta})$ as follows
\begin{equation}
\boldsymbol{Err}(T_{D,\theta}) = noise + bias^2 + var
\end{equation}
\vspace{-3mm}
\qquad \qquad \qquad \quad where
\vspace{-6.3mm}
\begin{align}
& noise = \boldsymbol{Err}(\phi_{\beta}) \notag \\
& bias^2 = [\phi_{\beta}(x) - \mathbb{E}_{D}\{T_{D,\theta}(x)\}]^2 \notag \\
& var = \mathbb{E}_{D}\{[\mathbb{E}_{D}(T_{D,\theta}(x)) - T_{D,\theta}(x)]^2\} \notag
\end{align}
The same decomposition can be conducted for the zero-one loss function in similar fashion and as mentioned in 
\cite{louppe2014understanding}, \cite{domingos2000decomposition}, 
\cite{james2003variance} and \cite{friedman1997zeroLoss},
both zero-one and squared loss functions decompositions yield same results. 
However, without assuming $D$ is normally distributed, bias-variance decomposition 
cannot be obtained for zero-one loss function as explicitly as done for squared loss \cite{louppe2014understanding}. 
\cite{kohavi1996bias} introduces another decomposition for zero-one loss function (included in section \ref{app:kohavi_decomp}), 
but still it remains to be unexplanatory compared to squared loss, thus, 
we explain the dynamics with using squared loss although 
the main focus of the paper remains to be on classification setting.
\vspace{2mm}
\\
\textbf{\emph{Extending Findings to Random Forest }}\\
In regression setting, random forest shares the same idea with classification prediction with soft voting. 
Random forest for regression setting can be written as
\begin{equation}
\boldsymbol{RF}_{D, \theta_{1},\theta_{2},..., \theta_{B}}(x) = \dfrac{1}{B}\sum_{b = 1}^{B}T_{D,\theta_{b}}(x)
\end{equation}
When we take the average prediction in this case equals to expectation in terms of training set, we get
\begin{align}\label{eq:mu}
\mathbb{E}_{D, \theta_{1}, \theta_{2},..., \theta_{B}}\{\boldsymbol{RF}_{D, \theta_{1},\theta_{2},..., \theta_{B}}(x) \} 
	& = \mathbb{E}_{D, \theta_{1}, \theta_{2},..., \theta_{B}}\{\dfrac{1}{B}\sum_{b = 1}^{B}T_{D,\theta_{b}}(x)\} \notag \\
	& = \dfrac{1}{B}\sum_{b=1}^{B}\mathbb{E}_{D,\theta_{b}}\{T_{D, \theta_{b}}(x)\}\notag \\
	& = \mu_{D,\theta}(x)
\end{align}
where $\mu_{D,\theta}(x)$ is the average prediction of all ensembled trees. Since $\theta$'s are random, 
independent and have the same distribution \cite{louppe2014understanding}, when we extend this finding bias of a random forest 
we can state that 
\begin{equation}\label{eq:random_forest_bias}
bias^2 = (\phi_{\beta}(x) - \mu_{D,\theta}(x))^2
\end{equation}
Intuitively, when we get the average prediction of all ensembled trees, 
we are able to consider $\mu_{D,\theta}(x)$ as one decision tree in the simplest terms and make this inference.
We can interpret equation (\ref{eq:random_forest_bias}) as the squared bias cannot be decreased with ensembling randomized models, 
namely, an ensemble of trees does not guarantee to have lower bias compared to only one tree \cite{friedman2001elements}.
Furthermore, the squared bias of an ensemble of randomized model equals 
the squared bias of that model in theory \cite{louppe2014understanding}. 
Although random forest is inadequate to propose any structure to decrease the generalization error so far regarding $noise$ 
and $bias^2$, it displays promising performance in reducing the last remaining part of the generalization error. 
Thus, we continue our exploration with the variance of Random Forest, yet, we need to define the correlation coefficient $\rho(x)$
before delving into variance since the correlation coefficient have a significant role in variance. 
For any two trees $T_{D,\theta'}$ and $T_{D,\theta''}$ trained with the same data $D$
and different growing parameters $\theta'$ and $\theta''$, we can define the correlation coefficient as follows
\begin{align}
	\rho(x) & 
	= \dfrac{\mathbb{E}_{D,\theta',\theta''}\{T_{D,\theta'}(x) T_{D,\theta''}(x)\} 
	- \mu_{D,\theta}^2(x)}{\sigma_{D,\theta}^2(x)}
\end{align}
We utilize the definition of Pearson's correlation coefficient and the property of $\theta'$ and $\theta''$ following 
the same distribution in the intermediate steps in section \ref{app:corr_coef}. 
$\sigma_{D, \theta}(x)$ is the variance of a single decision tree and 
associated with prediction variability stemming from the randomness of the set $D$ 
and randomness introduced with $\theta$ \cite{louppe2014understanding}.
Therefore, $\rho(x)$ represents the effect of randomization in the learning algorithm in general.
In our case, it is close to 1 when predictions of two Decision Trees 
are highly correlated and implies that randomization 
does not have a significant effect. 
On the other hand, if it is close to 0, trees are non-correlated and 
the prediction of trees are perfectly random in the sense that 
not dependent on the predictions of other trees. 
We can decompose the variance of Random Forest as follows:
\begin{align}\label{eq:decomp_var}
\mathbb{V}_{D, \theta_{1}, \theta_{2},..., \theta_{B}}\{\boldsymbol{RF}_{D, \theta_{1},\theta_{2},..., \theta_{B}}(x) \}  = \rho(x)\sigma^2_{D,\theta}(x) + \dfrac{1-\rho(x)}{B}\sigma^2_{D,\theta}(x)
\end{align}
We included derivations in detail in section \ref{app:var_decomp}. 
Increasing the number of ensembled trees $B$ will lower the latter expression in the equation and in the extreme 
case where $B \rightarrow \infty $, the variance of a random forest equals to $\rho(x)\sigma^2_{D,\theta}(x)$ 
and due to randomization in the algorithm $\rho(x) < 1$, 
thus, the variance of Random Forest is less than the variance of Decision Tree. 
This inference implies that the generalization error of Random Forest is less than the generalization error of Decision Tree. 
If the decision trees in the random forest are independent 
and consequently $\rho(x) \rightarrow 0$, the variance is reduced to $\sigma^2_{D,\theta}(x)/B$ 
which can be decreased with increasing $B$ as mentioned. 
Conclusively, Random Forest improves the performance of decision tree with decreasing variance and keeping bias unaffected.

\subsection{Interpretation}
In many cases, the main purpose of using a random forest model is its usefulness in performing predictions of a dependent variable
based on a set of explanatory variables. In addition, very often, to understand the processes under study we need to understand
which explanatory variables are more important and useful to make mentioned predictions. 
Random Forest allows to not only build an accurate model with reliable predictions, 
but also to provide variable importance measures which are very important in the process of interpreting the model and 
its prediction results. In this section while closely following \cite{louppe2013understanding} 
and \cite{gerard2016foresttour}, we will introduce general background to Mean Decrease Impurity 
and examine variable importance for totally randomized tree ensembles 
and then we will discuss these ideas' link to the Random Forest algorithm.

\subsubsection{Variable importance}
In order to rank the importance of the explanatory variables in classification problems using Random Forest 
we use two possible measures proposed by \cite{breiman2001random}; 
Mean Decrease Impurity (MDI) and Mean Decrease Accuracy (MDA). 
For MDI, we get the total decrease in node impurity from splitting 
with the given explanatory variable and we average that total over all trees. 
While closely following Louppe's notation \cite{louppe2013understanding}, 
we can define MDI as
\begin{equation}\label{eq:MDI}
	{MDI}( X_{j} ) = \frac{1}{B}\sum_{b=1}^{B}\sum_{t \in T_{b}}\mathds{1}(v(s_{t}) = X_{j})[ p(t)\Delta i(s_{t}, t)]
\end{equation}
where $ v(s_{t}) $ is the variable used in split $s_{t}$ and $ p(t) $ is the proportion $N_{t}/N$ of samples reaching $t$.
By definition $ MDI( X_{j} ) $ evaluates importance of a variable $ X_{j} $ for predicting $Y$ by 
adding up the weighted impurity decreases $p(t) \Delta i(s_{t}, t)$ for all nodes $t$ where $ X_{j}$ is used 
and averaging that sum over all $B$ trees in the Random Forest. 
We can use above definition for various impurity measure $i(t)$ including the Gini index and the information entropy.
Furthermore, in contemporary exercises, MDI is commonly used and 
also embedded into Random Forest algorithm in sklearn package \cite{scikit2011learn}. 
The second measure is known as Mean Decrease Accuracy (MDA). 
This measure assumes that if the explanatory variable is not relevant to the study of the problem, 
then rearranging its values should not demolish the model's prediction accuracy. 
MDA uses out-of-bag error estimate to calculate a variable's importance. 
In this case to measure the importance of the $X_{j}$ variable, we have to permute its values in the out-of-bag example and 
use these all observations in the tree. $ MDA( X_{j} )$ is counted with averaging the differences in 
out-of-bag error estimation after and before the permutation in all trees. 
Because of used permutations, 
this measure is sometimes called also as the permutation importance.

\subsubsection{Variable importance for totally randomized tree ensembles}
Let us now consider Mean Decrease Impurity as defined by equation \eqref{eq:MDI} and assume a set $U$ which is a set of categorical input variables and is equal to $\{X_1,...,X_p\}$ in addition assume categorical output $Y$. For the simplicity our impurity measure which we will use is the Shannon entropy. Let us assume that we created an infinitely large ensemble of trees which are totally randomized and fully developed. Let assume also that for creation of these trees we used training sample $D$ of $N$ joint observations of $X_1,...,X_p,Y$ independently taken from the joint distribution $P(X_1,...,X_p,Y)$. A totally randomized and fully developed tree as defined in \cite{louppe2013understanding} is a decision tree with partitioned every node $t$ using a variable $X_j$ chosen uniformly and randomly among those which were not yet used at the parent nodes of $t$
and where every $t$ is split into $|\Xb_j|$ sub-trees, where $\Xb$ is an image set. For these assumptions two following theorems have been proven.

\newtheorem{theorem}{Theorem}
\begin{theorem}
    The $MDI$ of $X_j \in U$ for $Y$ as computed with an infinite ensemble of fully developed totally randomized trees and an infinitely large training sample is:
    \begin{equation}
	{MDI}( X_{j} ) = \displaystyle \sum_{k=0}^{p-1}\frac{1}{C^k_p}\frac{1}{p-k} \displaystyle\sum_{O \in \Pb_{k}(U^{-j})} I(X_j;Y|O)
    \end{equation}
    
\end{theorem}
where $U^{-j}$ denotes the subset $U-\{X_j\}$, $\Pb_{k}(U^{-j})$ is the set of subsets of $U^{-j}$
of cardinality $k$ and $I(X_j;Y|O)$ is the conditional mutual information of $X_j$ and $Y$ given the variables $O$  as defined in \cite{kohavi1997importance}.

\begin{theorem}
	For any ensembles of fully developed trees in asymptotic learning sample size conditions (
	i.e. in the same conditions as those of Theorem 1), we have:
	\begin{equation}
	\displaystyle \sum_{j=1}^{p}{MDI}( X_{j} ) = I(X_1,...,X_p;Y)
     \end{equation}
	
\end{theorem}

\subsubsection{Relevant and irrelevant variable}

As it was done in \cite{kohavi1997importance} we define relevant and irrelevant variables. The variable $X_j$ for which exists at least one subset $O \subseteq U$ (possibly empty) s.t. $I(X_j;Y|O) > 0$ is a variable which is relevant to $Y$ with respect to $U$. In contrast, a variable which is irrelevant to $Y$ with respect to $U$ is a variable $X_i$ for which, for all 
 $O \subseteq U$, $I(X_i;Y|O) = 0$. Remark that if variable is relevant then by definition it can not be irrelevant in the same time, so $j \neq i$. Remark also that if $X_i$ is irrelevant to $Y$ with respect to $U$, then also by definition it is always irrelevant to $Y$ with respect to any subset of $U$. Louppe et al. using these definitions showed that below holds.

\begin{theorem}
	$X_i \in U$ is irrelevant to $Y$ with respect to $U$ if and only if its infinite sample size
	importance as computed with an infinite ensemble of fully developed totally randomized trees
	built on $U$ for $Y$ is 0.
\end{theorem}

\begin{theorem}
	 The infinite sample size importance of any variable $X_j \in U_R$ (where $U_R \subseteq U$ is the subset of all variables in $U$ which are relevant to $Y$ with respect to $U$) as computed with an infinite ensemble of fully developed totally randomized trees built on $U_R$ for $Y$ is the same as its importance computed in the same conditions by using all variables in $U$. It means:
\begin{equation}
	{MDI}( X_{j} ) = \displaystyle \sum_{k=0}^{p-1}\frac{1}{C^k_p}\frac{1}{p-k} \displaystyle\sum_{O \in \Pb_{k}(U^{-j})} I(X_j;Y|O) = 
	\displaystyle \sum_{l=0}^{r-1}\frac{1}{C^l_r}\frac{1}{r-l} \displaystyle\sum_{O \in \Pb_{l}(U^{-j}_R)} I(X_j;Y|O)
\end{equation}
\end{theorem}
 where $r$ is the number of relevant variables in $U_R$.
 \newline
 
Above two theorems inform us that that the variable importance obtained with an ensemble of totally randomized trees depends only on the relevant variables. It shows that importance of irrelevant variables is equal to 0 and accordingly it does not affect the importance of relevant variables. 
In practise, mentioned properties are desirable conditions for a reliable criterion for assessing the variable importance. As it should be expected, any noise should not have importance higher than zero and it should not influence on any other variable importance.

\subsubsection{Variable importance of non-totally randomized trees}

In the previous two subsections, analysis of variable importance do not directly relate to the Random Forest \cite{breiman2001random}, because we mention analysis of totally randomized ensembles of trees. As it is proved in \cite{kohavi1997importance} it is not true for Random Forest that variable is irrelevant if and only if its importance is equal to 0. It is also true that importance of variables is dependent on the number of irrelevant variables, so variable importance as obtained from totally randomized trees has properties that unfortunately Random Forest does not possess. Asymptotically, it is more appropriate to use totally randomized trees to assess the importance of variable. In the finite examples, where we have limited number of samples and limited number of trees, guiding the choice of the splitting variables is still a reliable strategy. In these cases, we can not measure $I(X_j;Y|O)$ for all $X_j$ and $O$. Hence, even if the resulting variable importance can be biased, it is pragmatic to promote those that seem to be the most promising. 

