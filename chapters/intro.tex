\section{Introduction}
\label{ch:intro}

Drawing conclusions from data and utilizing it to get predictions is a common 
practice in Economics alongside with many other fields. 
As a non-parametric estimation tool, Decision Trees attract attention in the literature,
yet we require robust methods to correctly identify patterns in the data and to obtain accurate predictions.
However, Decision Trees suffer from high variance \cite{friedman2001elements}. 
For prediction purposes, having high variance is a crucial problem, thus, several improvements were proposed.
These improvements are bootstrap aggregation, boosting and most importantly Random Forests, which is our focus in this paper. 
Fundamentally, the Random Forests is an ensemble of Decision Trees which are grown from randomly sampled data with 
randomly selected explanatory variables. 
Although, the Random Forest is also capable of regression, 
from our perspective, it gives an account of itself in a classification setting 
Therefore, we narrow our focus to the classification settings, 
meaning that the dependent variable in data is categorical.


We start by explaining the building block and the starting point of the Random Forest which is the Decision Tree.
We focused on the tree-building process with different splitting criteria rather than pruning, since
the Random Forest uses fully grown trees and does not prune them.
After delving into the variance issue, we describe one of the solutions, bootstrap aggregating (bagging) 
due to it being one of the main principles of the Random forest.
In the next section, we start with defining Random Forest 
and discuss the main root of randomness and mention the idea of 
Out of Bag sample which stems from the usage of bootstrapping.
There are two different class determination methods available in Random Forest. 
We start the Mathematical Explanation part by defining those voting processes. 
Although, we primarily use soft voting, understanding majority voting provides us with a better insight.
Then, we define a measure of the Decision Tree's fit and decompose it to 
understand the improvement of Random Forest over Decision Trees.
By exploiting bias-variance decomposition of this measure and expanding our findings to Random Forest, 
we exhibit the working principle of Random Forest.
In the next section, we examine variable importance as the Random Forest enables us to measure 
how important each independent variable is.
Finally, we applied the Random Forest algorithm to simulated and real data.
In the simulation study, we employed linear and non-linear data generation processes 
and compared Random Forest's performance with linear regression.
In the real data study, we used Titanic data \cite{titanicData} 
and compared Random Forest's performance with two boosting methods: Adaptive and Gradient Boosting.

In the literature, 
Decision Trees are covered by several published works including 
\cite{breiman1984classification} and \cite{James2013}.
As a solution to the problem of high variance in Decision Trees, 
Leo Breiman introduced bagging to add randomness to the model 
with randomly sampling the data\cite{breiman1996bagging}. 
Subsequently, the Random Forest is introduced as an extension including
randomness in the variable selection process \cite{breiman2001random}. 
In our paper, while \cite{friedman2001elements} being the main source regarding intuition,
Louppe provides us with a reassessment of published works and an insight into 
the mathematical aspects of the Random Forest\cite{louppe2014understanding}.
The bias-variance decomposition and the examination of 
the improvement of the Random Forest upon the Decision Tree makes use of
\cite{james2003variance}, \cite{domingos2000decomposition}, \cite{friedman1997zeroLoss}, and \cite{kohavi1996bias}. 
Louppe also clarifies the variable importance in \cite{louppe2013understanding} and 
\cite{kohavi1997importance} illuminates the variable selection process.

This paper is written as a term paper for the Research Module in Econometrics and Statistics and 
we would like to be graded as a group.
