\section{Introduction}
\label{ch:intro}

Drawing conclusions from data and utilizing it to get predictions is a common practice in Economics alongside with many other fields. 
As a non-parametric estimation tool, decision trees attract attention in the literature 
yet we require robust methods to correctly identify patterns in data and to obtain accurate predictions 
and decision trees suffer from high variance \cite{friedman2001elements}. 
For prediction purposes having high variance is a crucial problem, thus, several improvements were proposed 
such as bootstrap aggregation, boosting and most importantly Random Forests which is our focus in this paper. 
Fundamentally Random Forests are an ensemble of decision trees which are grown from randomly sampled data with 
randomly selected explanatory variables. 
Although Random Forest is also capable in regression type problems, 
from our perspective it gives an account of itself in classification setting and we scale down our focus to classification 
settings meaning that dependent variable in data is categorical. 


We start with explaining the decision trees since the decision tree is the building block and the starting point of Random Forest.
We concentrated on the tree-building process with differing splitting criteria rather than pruning 
as Random Forest uses fully grown trees and does not prune trees. 
After delving into the variance problem, we describe as one of the solutions, bootstrap aggregating (bagging) 
because Random forest utilizes the main principle of bagging, bootstrapped data. 
In the next section, we start with defining Random Forest and discuss the main root of randomness and mention the idea of 
Out of Bag sample which stems from the usage of bootstrapping.
There are two different class determination methods available theoretically in Random Forest. 
We start the Mathematical Explanation part with defining those voting processes. 
Although primarily we use soft voting, understanding majority voting provides us with a better insight.
Then we define a measure of the decision tree's fit and decompose it to understand the improvement of Random Forest over decision trees.
With exploiting bias-variance decomposition of that measure and expanding our findings to Random Forest, 
we exhibit the working principle of Random Forest.
In the next section, we examine variable importance as Random Forest enables us to measure 
how much each independent variable is important.
Finally, we applied the Random Forest algorithm to simulated and real data. 
In the simulation study, we employed linear and non-linear data generation processes and compared Random Forest's performance with linear regression.
In real data study, we used Titanic data \cite{titanicData} and compared Random Forest's performance with two boosting methods; Adaptive and Gradient Boosting.