\section{Appendix}
\subsection{Bayes Model explanation}
\label{app:bayes_model}
We identified Bayes Model in \autoref{eq:bayes_model} as such
\begin{align}
\phi_{\beta} & = \underset{c \,\in\, Y}{argmin} \; \mathbb{E}_{Y|X=x}\{L(Y,c) \} \notag\\
			 & = \underset{c \,\in\, Y}{argmin} \; P(Y \neq c \,| X = x) \notag\\
			 & = \underset{c \,\in\, Y}{argmax} \; P(Y = c \,| X = x)\notag
\end{align}

\subsection{Bias-variance decomposition of Squared Loss Function}
\label{app:bias_var_decomp}
We derived the result in \autoref{eq:decomp_squared_loss} as follows;
\begin{align}
\boldsymbol{Err}(T_{D, \theta}(x)) & = \mathbb{E}_{Y|X=x}\{(Y-T_{D,\theta}(x))^2\} \notag \\
							  & = \mathbb{E}_{Y|X=x}\{(Y -\phi_{\beta}(x) +\phi_{\beta}(x) -T_{D,\theta}(x))^2\} \notag \\
							  & = \mathbb{E}_{Y|X=x}\{(Y-\phi_{\beta}(x))^2\} + \mathbb{E}_{Y|X=x}\{(\phi_{\beta}(x)-T_{D,\theta}(x))^2\} \notag \\
							  &	\: + \underbrace{\mathbb{E}_{Y|X=x}\{2(Y-\phi_{\beta}(x))(\phi_{\beta}(x)-T_{D,\theta}(x))\}}_\text{$=0$ \ since $ \mathbb{E}_{Y|X=x}(Y-\phi_{\beta}(x)) = 0$ from $(18)$}  \notag \\
							  & = \underbrace{\mathbb{E}_{Y|X=x}\{(Y-\phi_{\beta}(x))^2\}}_\text{from $(20)$ equals to $\boldsymbol{Err}(\phi_{\beta}(x))$} + \mathbb{E}_{Y|X=x}\{(\phi_{\beta}(x)-T_{D,\theta}(x))^2\}  \notag \\
							  & = \boldsymbol{Err}(\phi_{\beta}(x)) + (\phi_{\beta}(x)-T_{D,\theta}(x))^2\notag
\end{align}
We adopted following steps in the futher derivations of bias-variance decomposition in \autoref{eq:decomp_squared_loss_cont}.
\begin{align}
& \mathbb{E}_{D}\{(\phi_{\beta}(x) - T_{D,\theta}(x))^2 \} \notag \\
& = \mathbb{E}_{D}\{(\phi_{\beta}(x) - \mathbb{E}_{D}\{T_{D,\theta}(x)\} + \mathbb{E}_{D}\{T_{D,\theta}(x)\} - T_{D,\theta}(x))^2\} \notag \\
& = \mathbb{E}_{D}\{(\phi_{\beta}(x) - \mathbb{E}_{D}\{T_{D,\theta}(x))^2\} 
	+ \mathbb{E}_{D}\{(\mathbb{E}_{D}\{T_{D,\theta}(x)\} - T_{D,\theta}(x))^2\} \notag \\
&\: \: + \mathbb{E}_{D}\{2(\phi_{\beta}(x) - \mathbb{E}_{D}\{T_{D,\theta}(x)\})(\mathbb{E}_{D}\{T_{D,\theta}(x)\} - T_{D,\theta}(x))\} \notag \\
& \text{since $\mathbb{E}_{D}\{\mathbb{E}_{D}\{T_{D,\theta}(x)\} - T_{D,\theta}(x)\} = \mathbb{E}_{D}\{T_{D,\theta}(x)\} - \mathbb{E}_{D}\{T_{D,\theta}(x)\} = 0$} \notag \\
& = \mathbb{E}_{D}\{(\phi_{\beta}(x) - \mathbb{E}_{D}\{T_{D,\theta}(x))^2\} 
	+ \mathbb{E}_{D}\{(\mathbb{E}_{D}\{T_{D,\theta}(x)\} - T_{D,\theta}(x))^2\} \notag \\
& = (\phi_{\beta}(x) - \mathbb{E}_{D}\{T_{D,\theta}(x))^2 + \mathbb{E}_{D}\{(\mathbb{E}_{D}\{T_{D,\theta}(x)\} - T_{D,\theta}(x))^2\}\notag
\end{align}

\subsection{Kohavi's decomposition of zero-one loss function}
\label{app:kohavi_decomp}
Kohavi proposed an alternative decomposition for zero-function \cite{kohavi1996bias}. 
Our notation differs in some extent from Kohavi's paper to be coherent with out prior findings. 
The zero-one loss function is defined as
\begin{align}
L(\phi_{\beta}(x), T_{D,\theta}(x)) = 1 - \delta(\phi_{\beta}(x), T_{D, \theta}(x)) \notag
\end{align}
where $\delta(\phi_{\beta}(x), T_{D, \theta}(x)) = 1$ if $\phi_{\beta}(x) = T_{D, \theta}(x)$ and 0 otherwise. The generalization error (misclassification rate in the paper) is defined and extented as
\begin{align}\label{eq:kohavi_eq}
\boldsymbol{Err}(T_{D, \theta}(x)) 
& = L(\phi_{\beta}(x), T_{D,\theta}(x) P(\phi_{\beta}(x), T_{D, \theta}(x)) \notag \\
& = \sum_{\phi_{\beta}(x), T_{D, \theta}(x)} [1 - \delta(\phi_{\beta}(x), T_{D, \theta}(x))] P(\phi_{\beta}(x), T_{D, \theta}(x)) \notag \\
& = 1 - \sum_{y \in Y} P(\phi_{\beta}(x) = T_{D, \theta}(x)) = y)
\end{align}
Even if in the Kohavi's paper it is stated that the last step is a 
simplified version of the extended Bayesian Formalism \cite{wolpert2018relationship}, 
there is not enough explanation to explicitly show the mathematical derivation nor the intuition. If we continue examine the \autoref{eq:kohavi_eq} we get the following decomposition;
\begin{align}\label{eq:kohavi_eq1}
\boldsymbol{Err}(T_{D, \theta}(x))
& = 1 - \sum_{y \in Y} P(\phi_{\beta}(x) = T_{D, \theta}(x)) = y) \notag\\
& = \sum_{y \in Y} -P(\phi_{\beta}(x) = T_{D, \theta}(x)) = y) + \sum_{y \in Y} P(\phi_{\beta}(x) = y) P(T_{D, \theta}(x)) = y) \notag\\
& + \sum_{y \in Y}[-P(\phi_{\beta}(x) = y) P(T_{D, \theta}(x)) = y) + \dfrac{1}{2}P(T_{D, \theta}(x) = y)^2 + \dfrac{1}{2}P(\phi_{\beta} = y)^2] \notag\\
& + [\dfrac{1}{2} - \dfrac{1}{2}\sum_{y \in Y} P(T_{D,\theta}(x)= y)^2] + [\dfrac{1}{2} - \dfrac{1}{2}\sum_{y \in Y} P(\phi_{\beta}(x)= y)^2]
\end{align}
Rearranging \autoref{eq:kohavi_eq1} yields
\begin{align}\label{eq:kohavi_eq2}
\boldsymbol{Err}(T_{D, \theta}(x))
& = \sum_{y \in Y} [P(T_{D,\theta}(x)=y)P(\phi_{\beta}(x) = y) - P(T_{D,\theta}(x) = \phi_{\beta}(x) = y)]\tag{$*$} \\
& + \dfrac{1}{2}\sum_{y \in Y} [P(T_{D,\theta}(x)=y) - P(\phi_{\beta}(x) = y)]^2\notag \\
& + \dfrac{1}{2}(1 - \sum_{y \in Y}P(T_{D, \theta}(x) = y)^2) + \dfrac{1}{2}(1 - \sum_{y \in Y}P(\phi_{\beta}(x) = y)^2)\notag \\
\end{align}
The $*$ term is the covariance between Bayes Model and decision tree which equals to zero since by construction Bayes Model cannot be dependent of any model. Then the decomposition becomes;
\begin{align}
\boldsymbol{Err}(T_{D, \theta}(x)) = \sum_{x} P(x)(\boldsymbol{Err}(\phi_{\beta}) + bias^2 + variance) \notag
\end{align}
where
\begin{align}
\boldsymbol{Err}(\phi_{\beta}) & = \dfrac{1}{2}(1 - \sum_{y \in Y}P(\phi_{\beta}(x) = y)^2)\tag{Bayes Error} \\
bias^2 & = \dfrac{1}{2}\sum_{y \in Y} [P(T_{D,\theta}(x)=y) - P(\phi_{\beta}(x) = y)]^2\notag \\
variance & = \dfrac{1}{2}(1 - \sum_{y \in Y}P(T_{D, \theta}(x) = y)^2) \notag
\end{align}
Since the outcome of Kohavi's decomposition is not as explanatory as our prior findings and without assuming a certain type of distribution for the sample \cite{louppe2014understanding}, we wanted to explain mathematical dynamics of random forest with using the squared loss function alongside zero-one loss function.

\subsection{Correlation Coefficient}
\label{app:corr_coef}
Using the definition of the Pearson's correlation coefficient and the property of $\theta'$ and $\theta''$ following the same distribution, we derived correlation coefficient as such
\begin{align}
\rho(x) & = \dfrac{\mathbb{E}_{D,\theta',\theta''}\{(T_{D,\theta'}(x) - \mu_{D,\theta'}(x))(T_{D,\theta''}(x) - \mu_{D,\theta''}(x))\} }				{\sigma_{D,\theta'}(x)\sigma_{D,\theta'}(x)}\notag \\
		& = \dfrac{\mathbb{E}_{D,\theta',\theta''}\{T_{D,\theta'}(x)T_{D,\theta''}(x) - T_{D,\theta'}(x)\mu_{D,\theta''}(x) - T_{D,								\theta''}(x)\mu_{D,\theta'}(x) + \mu_{D,\theta'}(x)\mu_{D,\theta''}(x))\} }
				{\sigma_{D,\theta}^2(x)} \notag \\
		& = \dfrac{\mathbb{E}_{D,\theta',\theta''}\{T_{D,\theta'}(x) T_{D,\theta''}(x)\} - \mu_{D,\theta}^2(x)}{\sigma_{D,\theta}^2(x)}\notag
\end{align}

With an alternative decomposition, we can state that $\rho(x)$ is non-negative 
\cite{louppe2014understanding} \cite{geurts2002contributions}. 
Being that said the findings we explored with decomposing the variance of the squared loss function are robust.

\subsection{Decomposition of Variance}
\label{app:var_decomp}
The derivation of \autoref{eq:decomp_var} is as follows;

\begin{align}
\mathbb{V}\{\boldsymbol{RF}\} & = \mathbb{V}_{D, \theta_{1}, \theta_{2},..., \theta_{B}}\{\boldsymbol{RF}_{D, \theta_{1},\theta_{2},..., \theta_{B}}(x) \}\notag \\
& = \mathbb{V}_{D, \theta_{1}, \theta_{2},..., \theta_{B}}\{\dfrac{1}{B}\sum_{b=1}^{B} T_{D,\theta_{b}}(x)\} \notag
\end{align}
Using $\mathbb{V}\{aX\} = a^2\mathbb{V}\{X\} = a^2 (\mathbb{E}\{X^2\} - \mathbb{E}\{X\}^2$, variance of random forest equals to

\begin{align}
\mathbb{V}\{\boldsymbol{RF}\} = \dfrac{1}{B}\left[\mathbb{E}_{D, \theta_{1}, \theta_{2},..., \theta_{B}}\{( \sum_{b=1}^{B} T_{D, \theta_{b}}(x))^2 \} - \mathbb{E}_{D, \theta_{1}, \theta_{2},..., \theta_{B}}\{( \sum_{b=1}^{B} T_{D, \theta_{b}}(x)) \}^2\right] \notag
\end{align}

Following Louppe's derivations, we can rewrite the variance as pairwise products of any two trees using parameters $\theta_{i}$ and $\theta_{j}$;

\begin{align}
\mathbb{V}\{\boldsymbol{RF}\} & = \dfrac{1}{B}[\mathbb{E}_{D, \theta_{1}, \theta_{2},..., \theta_{B}}\{\sum_{i,j} T_{D,\theta_{i}}(x) T_{D,\theta_{j}}(x)\} - (B \mu_{D,\theta}(x))^2] \notag
\end{align}
where $\mu_{D, \theta}$ is the average prediction of all ensembled trees and derived in \autoref{eq:mu}.
\begin{align}
\mathbb{V}\{\boldsymbol{RF}\} & = \dfrac{1}{B^2}\left[\sum_{i,j} \mathbb{E}_{D, \theta_{i}, \theta_{j}}\{T_{D,\theta_{i}}(x) T_{D,\theta_{i}}(x)\} - B^2 \mu_{D,\theta}^2(x)\right]\notag \\
& = \dfrac{1}{B^2}\left[B\mathbb{E}_{D,\theta}\{T_{D,\theta}(x)^2\} + (B^2 - B) \mathbb{E}_{D, \theta_{i}, \theta_{j}}\{T_{D,\theta_{i}}(x) T_{D,\theta_{i}}(x)\} - B^2 \mu_{D,\theta}^2(x)\right]\notag \\
& = \dfrac{1}{B^2}\left[B(\sigma_{D,\theta}^2(x) + \mu_{D,\theta}^2(x)) + (B^2 - B) (\rho(x)\sigma_{D,\theta}^2(x) + \mu_{D, \theta}^2(x)) - B^2 \mu_{D,\theta}^2(x)\right] \notag \\
& = \dfrac{\sigma_{D,\theta}^2(x)}{B} + \rho(x)\sigma_{D,\theta}^2(x) - \rho(x)\dfrac{\sigma_{D,\theta}^2(x)}{B}\notag \\
& = \rho(x)\sigma_{D,\theta}^2(x) + \dfrac{1 - \rho(x)}{B}\sigma_{D,\theta}^2(x) \notag
\end{align}