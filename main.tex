%template Master Thesis 
%University of Bonn Master of Life Science Informatics
% arara: pdflatex: { synctex: on }

\documentclass[twoside, 12pt,  footinclude=true,  headinclude=true,  cleardoublepage=empty]{scrbook}

\usepackage[utf8]{inputenc}
\usepackage [english] {babel} 

\usepackage[]{biblatex}
\addbibresource{references.bib}

\usepackage{lipsum}
\usepackage[linedheaders,parts,pdfspacing]{classicthesis}
\usepackage{amsmath}
\usepackage{amsthm}
\usepackage{booktabs}
\usepackage{graphicx}
\usepackage{float}
\usepackage{indentfirst}
\usepackage [T1]{fontenc}
\usepackage{listings}
\usepackage{color}
\usepackage{multirow}
\usepackage{tikz}
\usepackage[toc,page]{appendix}
\usepackage{MnSymbol}
\usepackage{longtable}
\usepackage{graphicx}
\usepackage{subcaption}
\usepackage{mathtools} 
\usepackage{enumerate}
\usepackage{csquotes}
\usepackage{amsmath}
\usepackage{hyperref}
\usepackage{acro}
\usepackage[a4paper,includeall,bindingoffset=20mm,margin=2cm,marginparsep=0cm,marginparwidth=0cm]{geometry}
\usepackage[font={footnotesize,it}, labelfont=bf]{caption}

\DeclareAcronym{AD}{
	short = AD ,
	long  = Alzheimer's disease
}
\DeclareAcronym{API}{
	short = API ,
	long  = Application Programming Interface
}
\DeclareAcronym{BioPAX}{
	short = BioPAX,
	long  = Biological Pathway Exchange Language
}
\DeclareAcronym{BEL}{
	short = BEL,
	long  = Biological Expression Language
}
\DeclareAcronym{BELIEF}{
	short = BELIEF,
	long  = Biological Expression Language Information Extraction Workflow
}
\DeclareAcronym{BRENDA}{
	short = BRENDA,
	long  = Braunschweig Enzyme Database
}
\DeclareAcronym{ChEBI}{
	short = ChEBI,
	long  = Chemical Entities of Biological Interest
}
\DeclareAcronym{CI}{
	short = CI,
	long  = Continuous Integration
}
\DeclareAcronym{CSV}{
	short = CSV,
	long  = Comma Separated Values
}
\DeclareAcronym{DL}{
	short = DL,
	long  = Descriptive Logic
}
\DeclareAcronym{eQTL}{
	short = eQTL,
	long  = Expression Quantitative Trait Loci
}
\DeclareAcronym{eSNPO}{
	short = eSNPO,
	long  = eQTL Single Nucleotide Polymorphism Ontology
}
\DeclareAcronym{FCS}{
	short = FCS,
	long  = Functional Class Scoring
}
\DeclareAcronym{GML}{
	short = GML,
	long  = Graph Markup Language
}
\DeclareAcronym{GO}{
	short = GO,
	long  = Gene Ontology
}
\DeclareAcronym{GraphQL}{
	short = GraphQL,
	long  = Graph Query Language
}
\DeclareAcronym{GRP}{
	short = GRP,
	long  = Gene Set File Format
}
\DeclareAcronym{GSEA}{
	short = GSEA,
	long  = Gene Set Enrichment Analysis
}
\DeclareAcronym{HGNC}{
	short = HGNC,
	long  = HUGO Gene Nomenclature Committee
}
\DeclareAcronym{HTML}{
	short = HTML,
	long  = HyperText Markup Language
}
\DeclareAcronym{HUGO}{
	short = HUGO,
	long  = Human Genome Organization
}
\DeclareAcronym{IMI}{
	short = IMI,
	long  = International Medicine Initiative
}
\DeclareAcronym{INDRA}{
	short = INDRA,
	long  = Integrated Dynamical Reasoner and Assembler
}
\DeclareAcronym{IRI}{
	short = IRI,
	long  = Internationalized Resource Identifier
}
\DeclareAcronym{JGIF}{
	short = JFIG,
	long  = JSON Graph Interchange Format
}
\DeclareAcronym{JSON}{
	short = JSON,
	long  = JavaScript Object Notation
}
\DeclareAcronym{JSONLD}{
	short = JSON-LD,
	long  = JSON Linked Data
}
\DeclareAcronym{KEGG}{
	short = KEGG,
	long  = Kyoto Encyclopedia of Genes and Genomes
}
\DeclareAcronym{MeSH}{
	short = MeSH,
	long  = Medical Subject Headings
}
\DeclareAcronym{NDEx}{
	short = NDEx,
	long  = Network Data Exchange
}
\DeclareAcronym{NeuroMMSig}{
	short = NeuroMMSig,
	long  = Multimodal Mechanistic Signatures for Neurodegenerative Diseases
}
\DeclareAcronym{NPA}{
	short = NPA,
	long  = Network Perturbation Amplitude
}
\DeclareAcronym{OBO}{
	short = OBO,
	long  = Open Biomedical Ontology
}
\DeclareAcronym{OLS}{
	short = OLS,
	long  = Ontology Lookup Service
}
\DeclareAcronym{ORA}{
	short = ORA,
	long  = Over Representation Analysis
}
\DeclareAcronym{OWL}{
	short = OWL,
	long  = Web Ontology Language
}
\DeclareAcronym{PD}{
	short = PD,
	long  = Parkinson's disease
}
\DeclareAcronym{PT}{
	short = PT,
	long  = Pathway Topology
}
\DeclareAcronym{PTSD}{
	short = PTSD,
	long  = Post-traumatic Stress Disorder
}
\DeclareAcronym{miRNA}{
	short = miRNA,
	long  = Micro-Ribonucleic Acid
}
\DeclareAcronym{mRNA}{
	short = mRNA,
	long  = Messenger Ribonucleic Acid
}
\DeclareAcronym{RCR}{
	short = RCR,
	long  = Reverse Causal Reasoning
}
\DeclareAcronym{RDF}{
	short = RDF,
	long  = Resource Description Format
}
\DeclareAcronym{RDFS}{
	short = RDFS,
	long  = Resource Description Format Schema
}
\DeclareAcronym{REST}{
	short = REST,
	long  = Representational State Transfer
}
\DeclareAcronym{RNA}{
	short = RNA,
	long  = Ribonucleic acid
}
\DeclareAcronym{SBML}{
	short = SBML,
	long  = Systems Biology Markup Language
}
\DeclareAcronym{SIF}{
	short = SIF,
	long  = Simple Interaction Format
}
\DeclareAcronym{SPARQL}{
	short = SPARQL,
	long  = SPARQL Protocol and RDF Query Language
}
\DeclareAcronym{SQL}{
	short = SQL,
	long  = Structured Query Language
}
\DeclareAcronym{SNP}{
	short = SNP,
	long  = Single-Nucleotide Polymorphism
}
\DeclareAcronym{SST}{
	short = SST,
	long  = Sampling of Spanning Trees
}
\DeclareAcronym{TBI}{
	short = TBI,
	long  = Traumatic Brain Injury
}
\DeclareAcronym{UBERON}{
	short = UBERON,
	long  = Uber Anatomy Ontology
}
\DeclareAcronym{UniProt}{
	short = UniProt,
	long  = Universal Protein Resource
}
\DeclareAcronym{XML}{
	short = XML,
	long  = eXtensible Markup Language
}
\DeclareAcronym{XMLS}{
	short = XMLS,
	long  = eXtensible Markup Language Schema
}
\DeclareAcronym{XGMML}{
	short = XGMML,
	long  = eXtensible Graph Markup and Modeling Language
}

\title{Master Thesis}
\author{Charles Tapley Hoyt}
\date{\today}
\begin{document}
	\begin{titlepage}
		\centering
		Bonn-Aachen International Center for Information Technology (B-IT)
		
		University of Bonn
		
		 Master Programme in Life Science Informatics
		
		\vspace{1in}
		 {\Large \bfseries Master's Thesis}
		\vspace{1in}
		
		{\LARGE \bfseries PyBEL: a Computational Framework for Biological Expression Language}
		\vspace{1in}
		
		{\large Submitted by}
		
		{\LARGE Charles Tapley Hoyt\par}
		
		\vspace{1in}
		
			First Supervisor: Prof. Dr. Martin Hofmann-Apitius
			\par
			Second Supervisor: Prof. Dr. Thomas Schultz
			\par
			Internal Supervisor: Christian Ebeling
			
		\vfill
		In collaboration with the Fraunhofer Institute for Algorithms and Scientific Computing (SCAI)
		\begin{flushleft}
			\today
		\end{flushleft}
		
	\end{titlepage}
	
%	% Add blank page
%	\newpage
%	\thispagestyle{empty}
%	\mbox{}
	
	% /frontmatter -> Turn on roman numbering for the following content and turns off normal numbering
	
	\frontmatter


\tableofcontents


\chapter*{Abstract}
The quantity of data, information, and knowledge in the biomedical domain is increasing at an unprecedented rate — with no signs of deceleration. Even with the assistance of information retrieval technologies, it is overwhelming, if not impossible, for individuals or groups of researchers to be knowledgeable of the state-of-the-art in any but an incredibly specific topic. Besides their obvious increases in volume and velocity, data are also increasing in variety as multi-modal and multi-scale experiments grow more important in the investigation of complex diseases. As experiments' complexities grow, so does the intellectual and temporal burden of analysis and interpretation. 

The ability to reason over the wealth of knowledge from both structured and unstructured sources to generate and prioritize hypotheses in order to automatically interpret new data sets would provide a huge relief to this burden. 

Developing systematic and reproducible methods first requires the formalization and assembly of knowledge in a computable form. As an aside, many techniques and methodologies in bioinformatics are biased towards the study of cancer biology and focus on data and knowledge at the molecular level. In this modeling strategy, often called the bottom-up approach, network and mathematical models are validated against the literature and experiments. 

As we foray into the assembly of knowledge pertaining to new disease areas and associated clinical indications, we find much more focus on the process level and phenotypic level. Because the links between genetics, molecular mechanisms, phenotypes, and clinical measurements are much less clear, they also require the top-down approach to modeling, which first focuses on the larger scales.  While most modeling languages and data formats for assembling knowledge are insufficient, the \ac{BEL} possesses the unique faculty to capture this multi-scale knowledge. It has the potential to serve as a semantic integration platform on which the data measured across scales can be integrated and analyzed. 

The purpose of this work is to outline the first steps taken towards the building of an automatic interpretation and hypothesis generation machine. The contents of this thesis describe the framework built to parse and manipulate the knowledge assemblies encoded in \ac{BEL}, which enables \ac{BEL} to act as a semantic integration layer for heterogeneous data and knowledge sources, the development of a framework for automatic integration of relevant knowledge from structured sources, and the development of schema-free analytical techniques to generate data-driven hypothesis.

% /frontmatter -> Turn on normal numbering 
\mainmatter

\section{Introduction}
\label{ch:intro}

Lorem ipsum dolor sit amet, consetetur sadipscing elitr, sed diam nonumy eirmod tempor invidunt ut 
labore et dolore magna aliquyam erat, sed diam voluptua. At vero eos et accusam et justo duo 
dolores et ea rebum. Stet clita kasd gubergren, no sea takimata sanctus est Lorem ipsum dolor sit amet.
Lorem ipsum dolor sit amet, consetetur sadipscing elitr, sed diam nonumy eirmod tempor invidunt ut labore
et dolore magna aliquyam erat, sed diam voluptua. At vero eos et accusam et justo duo dolores et ea rebum.
Stet clita kasd gubergren, no sea takimata sanctus est Lorem ipsum dolor sit amet.

\chapter{Motivation \& Outline}
\label{ch:motivation_goal}

\section{Motivation}

The overarching purpose of this master's thesis is to outline the first steps taken towards building a generally reusable, automatic interpretation and hypothesis generation machine.

While many analyses, including those previously referenced, can successfully aid scientists in interpretation of data, each are developed with specific data sets, knowledge assemblies, or application scenarios in mind. Those that were developed using schemata that lacked the generalized multi-scale and multi-modal (schema-free) integration enabled by BEL could have potentially disregarded relevant and important knowledge. 

\section{Outline}

The first section of this thesis describes the PyBEL, the framework built to parse and manipulate BEL Script and resulting knowledge assemblies. The following section describes the development the Bio2BEL data integration framework and the beginning of a cross-scale data integration project similar to Pathway Commons. The final section describes the development of algorithms for analyzing the robustness of knowledge assemblies, preprocessing techniques, and ultimately, proposes a reusable, general, schema-free analytical technique that generates hypotheses with knowledge-driven analyses of data.


% \include{pybel_core}

% \include{bio2bel}

% \include{pybel_tools}

\section{Conclusion and outlook}

We examined and applied one of the infamous machine learning algorithms, Random Forest. 
We started with explaining the Decision Trees and the room for improvement in it. 
As solutions of high-variance in Decision Trees, bagging, boosting and Random Forest are developed. 
Random Forest shows better predictions can be achieved with introducing randomness into the picture.
That randomness provides us with a decorrelated ensemble of trees and 
the increase in the number of trees yields lower error considering prediction purposes.
With using Random Forest, we are able to assess the importance of every variable and draw conclusions about data.
We explained the intuition of Random Forest in detail and included mathematical clarification. 
Finally, we applied Random Forest on both simulated and real data and compared with various methods. 
Considering results, Random Forest appears to be employed in the future as well, 
thus, a better understanding of the idea and the dynamics can give us a chance to improve. 
This paper aims to introduce the idea in detail and essentially a review and a showcase of the Random Forest algorithm. 

\printbibliography
	
\backmatter

\chapter*{Declaration}
I hereby certify that this material is my own work, that I used only those sources and resources referred to in the thesis, and that I have identified citations as such.
		
\vspace{0.3in}

\noindent Bonn, \today

\vspace{1in}

\noindent Charles Tapley Hoyt

\end{document}
