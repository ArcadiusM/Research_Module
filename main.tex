%template Master Thesis 
%University of Bonn Master of Life Science Informatics
% arara: pdflatex: { synctex: on }

\documentclass[ twoside=false, 12pt,  footinclude=true,  headinclude=true,  cleardoublepage=empty]{scrbook}

\usepackage[utf8]{inputenc}
\usepackage [english] {babel} 

\usepackage[]{biblatex}
\addbibresource{references.bib}

\usepackage{lipsum}
\usepackage[linedheaders,parts,pdfspacing]{classicthesis}
\usepackage{amsmath}
\usepackage{amsthm}
\usepackage{booktabs}
\usepackage{graphicx}
\usepackage{float}
\usepackage{indentfirst}
\usepackage [T1]{fontenc}
\usepackage{listings}
\usepackage{color}
\usepackage{multirow}
\usepackage{tikz}
\usepackage[toc,page]{appendix}
\usepackage{MnSymbol}
\usepackage{longtable}
\usepackage{graphicx}
\usepackage{subcaption}
\usepackage{mathtools} 
\usepackage{enumerate}
\usepackage{csquotes}
\usepackage{amsmath}
\usepackage{hyperref}
\usepackage{acro}
\usepackage[a4paper,includeall,margin=2cm,marginparsep=0cm,marginparwidth=0cm,left=3cm,right=2cm,top=2cm,bottom=2cm]{geometry}
\usepackage[font={footnotesize,it}, labelfont=bf]{caption}

\setlength{\parskip}{10pt} % Abstand zwischen den Absätzen
\setlength{\parindent}{0pt} %Erstzeileneinzug 
\setcounter{secnumdepth}{3}
\setcounter{tocdepth}{4}

\DeclareAcronym{AD}{
	short = AD ,
	long  = Alzheimer's disease
}

\title{Master Thesis}
\date{\today}
\begin{document}
	\begin{titlepage}
		\centering
		University of Bonn
		
		 Master Programme in Economics
		\vspace{1in}
		\vspace{1in}
		
		{\LARGE \bfseries  Random Forest for Classification Problems}
		\vspace{1in}
		
		{\large Submitted by}
		
		{\LARGE Burak Balaban \par
				Arkadiusz Modzelewski\par
				Raphael Redmer\par}
		
		\vspace{1in}
		
			Supervisor: Prof. Dr. Dominik Liebl
			
		\vfill
		
		\begin{flushleft}
			\today
		\end{flushleft}
		
	\end{titlepage}
	
%	% Add blank page
%	\newpage
%	\thispagestyle{empty}
%	\mbox{}
	
	% /frontmatter -> Turn on roman numbering for the following content and turns off normal numbering
	
	\frontmatter

\tableofcontents

\chapter*{Abstract}
As a non-parametric estimation tool, decision trees attract attention in the economics literature. Yet, decision trees suffer from high variance and, for prediction purposes higher variance seems to be a crucial problem, thus, several improvements were proposed such as bootstrap aggregation, boosting and most importantly random forests. In this project, while the main focus is being on the random forest, The elements of statistical learning by \cite{friedman2001elements} \cite{varian2014big} \cite{maimon2005data}, \cite{louppe2014understanding} and as expected \cite{breiman2001random} are the main literature that will be utilized in this project.

To explain the concept of random forests in full extent, primarily decision trees should be discussed. Exploiting the main idea and struggles with bias-variance trade-off, random forests' importance can be emphasized as a more stable prediction tool \cite{maimon2005data}. Conceptual comparison of random forests with bagging and boosting can deliver a better understanding of its unique features as \cite{lee2019bootstrap} shows in a similar fashion. To get a further understanding, random forests’ estimation process can be mathematical explained \cite{biau2012analysis} and likewise, examining the consistency of estimator and showing the properties can be included \cite{breiman2004consistency}, \cite{denil2014narrowing}. Also, variable importance in the tree growing process is another area that needs to be delved into \cite{ishwaran2007variable} and \cite{louppe2013understanding}.



% /frontmatter -> Turn on normal numbering 
\mainmatter

\section{Introduction}
\label{ch:intro}

Drawing conclusions from data and utilizing it to get predictions is a common practice in Economics alongside with many other fields. 
As a non-parametric estimation tool, decision trees attract attention in the literature 
yet we require robust methods to correctly identify patterns in data and to obtain accurate predictions 
and decision trees suffer from high variance \cite{friedman2001elements}. 
For prediction purposes having high variance is a crucial problem, thus, several improvements were proposed 
such as bootstrap aggregation, boosting and most importantly Random Forests which is our focus in this paper. 
Fundamentally Random Forests are an ensemble of decision trees which are grown from randomly sampled data with 
randomly selected explanatory variables. 
Although Random Forest is also capable in regression type problems, 
from our perspective it gives an account of itself in classification setting and we scale down our focus to classification 
settings meaning that dependent variable in data is categorical. 


We start with explaining the decision trees since the decision tree is the building block and the starting point of Random Forest.
We concentrated on the tree-building process with differing splitting criteria rather than pruning 
as Random Forest uses fully grown trees and does not prune trees. 
After delving into the variance problem, we describe as one of the solutions, bootstrap aggregating (bagging) 
because Random forest utilizes the main principle of bagging, bootstrapped data. 
In the next section, we start with defining Random Forest and discuss the main root of randomness and mention the idea of 
Out of Bag sample which stems from the usage of bootstrapping.
There are two different class determination methods available theoretically in Random Forest. 
We start the Mathematical Explanation part with defining those voting processes. 
Although primarily we use soft voting, understanding majority voting provides us with a better insight.
Then we define a measure of the decision tree's fit and decompose it to understand the improvement of Random Forest over decision trees.
With exploiting bias-variance decomposition of that measure and expanding our findings to Random Forest, 
we exhibit the working principle of Random Forest.
In the next section, we examine variable importance as Random Forest enables us to measure 
how much each independent variable is important.
Finally, we applied the Random Forest algorithm to simulated and real data. 
In the simulation study, we employed linear and non-linear data generation processes and compared Random Forest's performance with linear regression.
In real data study, we used Titanic data \cite{titanicData} and compared Random Forest's performance with two boosting methods; Adaptive and Gradient Boosting.

In the literature, 
decision trees are covered by several published works including 
\cite{breiman1984classification} and \cite{James2013}.
As a solution to the problem of high variance in decision trees, 
Leo Breiman introduced bagging to add randomness to the model 
with randomly sampling the data\cite{breiman1996bagging}. 
Subsequently, Random Forest is introduced as an extention with 
also embedding randomness in the variable selection process\cite{breiman2001random}. 
In our paper, while \cite{friedman2001elements} being the main source regarding intuition,
Louppe provides us with a re-assesment of published works and insight in 
mathematical aspects of Random Forest\cite{louppe2014understanding}.
The bias-variance decomposition and examination of improvement of Random Forest upon decision trees utilizes 
\cite{james2003variance}\cite{domingos2000decomposition}\cite{friedman1997zeroLoss}\cite{kohavi1996bias}. 
Louppe clarifies also the variable importance in \cite{louppe2013understanding} and 
\cite{kohavi1997importance} illuminates the variable selection process.

This paper is written as a term paper for Research Module in Econometrics and Statistics and 
we would like to be graded as a group.


\chapter{Decision tree}
Lorem ipsum dolor sit amet, consetetur sadipscing elitr, sed diam nonumy eirmod tempor invidunt ut labore
et dolore magna aliquyam erat, sed diam voluptua. At vero eos et accusam et justo duo dolores et ea rebum.

%%%%%%%%%%%%%%%%%%%%%%%%%%%%%%%
\section{Main idea}
The Decision Tree is a non-parametric supervised learning method used for classification and regression.
It predicts the response with a set of if-then-else decision rules derived from the data.
The deeper the tree, the more complex the decision rules and the closer the model fits the data.
The decision tree builds classification or regression models in the form of a tree structure. 
Each node in the tree further partions the feature space into smaller and smaller subsets 
while at the same time an associated decision tree is incrementally developed.
The final result is a tree with decision nodes and terminal nodes. 
A decision node has two or more branches.
Leaf node represents a classification or decision. 
The topmost decision node in a tree which corresponds to the best predictor is called the root node.
Decision trees can handle both categorical and numerical data.

An example of such a tree is depicted below in figure \ref{Fig:decision_tree_example}.

\begin{figure}[H]
    \captionsetup{format=plain}
    \makebox[\textwidth]{\includegraphics[width=120mm]{decision_tree_example.png}}
    \caption{Given a data set with two features height and weight, and gender as the target variable, 
             this example tree stratisfies the two-dimensional feature space into three distinct subset each 
             represented by the terminal nodes at the bottom.
             The stratification occurs at the two deciding nodes depending either on whether its height is above 180 cm 
             and or its weight  is above 80kg.
             }
    \label{Fig:decision_tree_example}
\end{figure}

%%%%%%%%%%%%%%%%%%%%%%%%%%%%%%%
\section{Tree Building Process}
This chapter describes the CART algorithm for tree building as specified in \cite{breiman1984classification}.
The basic idea of tree growing is to choose a split among all the possible splits at each node
so that the resulting child nodes are the “purest”. In this algorithm, only univariate splits are
considered. That is, each split depends on the value of only one predictor variable. All
possible splits consist of possible splits of each predictor.

A tree is grown starting from the root node by repeatedly using the following steps on each
node (also called binary splitting)

\begin{itemize}
    \item[(i)] \textbf{Find best split \(s\) for each feature \(X_{m}\):}
    For each feature \(X_{m}\), there exist \(K-1\)-many potiential splits whereas \(K\) is the number of different values for the respective feature.
    Evaluate each value \(X_{m,i}\) at the current node \(t\) as a candidate split point (for \(x \in X_{m}\), if \(x \leq X_{m,i}=s\),
    then \(x\) goes to left child node \(t_{L}\) else to right child node \(t_{R}\)).
    The best split point is the one that maximize the splitting criterion \(\ \Delta i(s,t) \) the most when the node is split according to it.
    The different splitting criteria will be covered in the next chapter.
    \item[(ii)] \textbf{Find the node’s best split:} Among the best splits for each feature from Step (i) find the one \(s^{*}\), which maximizes the splitting criterion \(\Delta i(s,t)\).
    \item[(iii)] \textbf{Satisfy stopping criterion:} Split the node \(t\) using best node split \(s^{*}\) from Step (ii) and 
    repeat from Step (i) until stopping criterion is satisfied. 
\end{itemize}

%%%%%%%%%%%%%%%%%%%%%%%%%%%%%%%
\subsection{Splitting criteria}
Since we are only concerned with classification, \(Y\) is categorical. The original CART algorithm uses Gini and Twoing as 
purity measures for the splitting criterion. However, implementations of the algorithm such as Python's sklearn package
also contain entropy and misclassification rate as measures of impurity.

For a give learning sample \(L\) for a \(J\) class problem, let \(N_{j}\) be the number of instances \( \{x,y\}  \)
belonging to in class \(j\).

In node \(t\), let \(N(t)\) be the total number of instances with \( \{x,y\} \in t \) and \( N_{j}(t) \) the number of class \(j\)
cases in \(t\). The proportion of the class \(j\) instances in the sample \(L\) falling into \(t\) is  \( N_{j}(t) / N_{j} \).
For a given set of priors, \( \pi(j) \) is interpreted as the probability that an instance belongs to class \(j\).

At node \(t\) let the probabilities \(p(j,t)\), \(p(t)\) and \(p(j|t)\) be estimated by using
Thus, let 

\begin{equation}
    p(j,t) = \frac{ \pi(j)N_{j}(t) }{ N_{j} }
\end{equation}

be the estimate for the probability that na instance will both be in class \(j\) and fall into node \(t\).
Therefore, the estimate for the probability that any instance falls into node \(t\) is defined by

\begin{equation}
    p(t) = \sum_{j} p(j,t),
\end{equation}

The estimate \( p(t) \) for the probability that an instance belongs to class \(j\) given that it falls into node \(t\) is defined by

\begin{equation}
    p(j|t) = \frac{ p(j,t)}{ p(t) } = \frac{ p(j,t) }{ \sum_{j} p(j,t) }.
\end{equation}

It holds that the conditional probability \(p(j|t)\) must satisfy

\begin{equation}
    \sum_{j} p(j|t)  = 1
\end{equation}

Let \(i(t)\) be an impurity measure evaluated at note \(t\). Then, the decrease of impurity (i.e. the splitting criterion) is defined as

\begin{equation}
    \Delta i(s,t) = i(t) - p_{L} i(t_{L}) - p_{R} i(t_{R}),
\end{equation}

where \(p_{L}\) and \(p_{R}\) are probabilities of sending a case to the left child node \(t_{L}\) and to the
right child node \(t_{R}\) respectively. 
They are estimated as \( p_{L} = p(t_{L}) / p(t) \) and \( p_{R} = p(t_{R}) / p(t) \).

As already stated abobe, the goal is to maximize \(\Delta i(s,t)\)
In the following, different measures for impurity will be presented.

\textbf{Gini Measure}

The Gini impurity measure is defined as 


\begin{equation}
    i(t) = \sum_{j} p(j|t) (1 - p(j|t)) = 1 - \sum_{j=1}^{J} p_{j}^{2}
\end{equation}

The intuition behind this measure is to assign nodes for which its probabilities are more skewed towards a particular group a higher value.
Conversely, if a node has more balanced distribution, then \(i(t)\) will turn out to be lower.
For example, in the case of \( J=2 \), \(i(t)\) will be maximized with \( p(j|t) = 0.5 \) for \(j = 1, 2\).


\textbf{Information Entropy}

The Entropy measure from information theory is defined as

\begin{equation}
    i(t) = \sum_{j} p(j|t) log(p(j|t))
\end{equation}

which measures the average rate at which information is produced by a stochastic source of data .
Thus, it can also be used for measuring impurity.

\newpage

\textbf{Rate of Misclassification}

The rate of misclassification is defined as 

\begin{equation}
    i(t) = 1 - \max_{0 < j \leq J} p(j|t) .
\end{equation}

which measure the proportion of instances node \(t\) not belonging to the dominant group in \(t\).


%%%%%%%%%%%%%%%%%%%%%%%%%%%%%%%%%%%%%
\section{Bias-variance trade-off}
Lorem ipsum dolor sit amet, consetetur sadipscing elitr, 
sed diam nonumy eirmod tempor invidunt ut labore et dolore magna aliquyam erat, sed diam voluptua.
At vero eos et accusam et justo duo dolores et ea rebum. Stet clita kasd gubergren,
no sea takimata sanctus est Lorem ipsum dolor sit amet.

\subsection{Bagging and boosting}
Lorem ipsum dolor sit amet, consetetur sadipscing elitr, 
sed diam nonumy eirmod tempor invidunt ut labore et dolore magna aliquyam erat, sed diam voluptua.
At vero eos et accusam et justo duo dolores et ea rebum. Stet clita kasd gubergren,
no sea takimata sanctus est Lorem ipsum dolor sit amet.

\section{Random Forest}
Decision trees, which we mentioned in the previous section, have been used for a long time. Deployment of decision trees is visible in simple situations and also in more complex scientific or real life and industrial affairs. The recent popularity of decision trees is due to work presented by Breiman between 1996 and 2004 that ensamples of different decision trees can get a meaningful improvement in accuracy in classification problems and other common learning tasks such as regression. As the unit in this procedure is based on decision tree and includes an injection of randomness, this method is known as a random forest. 

\subsection{Main idea and illustration}
An ensemble of randomly trained decision trees, so in other words random decision forest was defined by Breiman in his work ,,Random Forests”  from 2001 as follows:


% \theoremstyle{definition} % amsthm only
\newtheorem{theorem}{Theorem}
\begin{theorem}
A random forest is a classifier consisting of a collection of tree-structured classifiers \{${g(\textbf{x},\Theta_{k})}, k = 1,2,...$\} where the \{$\Theta_{k}$\} are independent identically
distributed random vectors and each tree casts a unit vote for the most popular class at input $\textbf{x}$ .
\end{theorem}

\subsection{Randomness in the model}
The main aspect of a random decision forest model is an injection of randomness which allows to have all unit trees different from the others. Two key concepts that makes decision forest "random" are:
\begin{enumerate}
\item Random sampling of training data points when building trees
\item Random subsets of features considered when splitting nodes
\end{enumerate}
In practise, random sampling of training observations is done by using bootstrapping. Bootstrapping for random forests means that every single decision tree learns from a random sample which is drawn with replacement. The second important injection of randomness into model is taking into consideration only a subset of all the variables, when splitting each node in every unit decision tree. It means that number of variables for splitting the node has to be chosen and for classification recommended number is $\lfloor{\sqrt{n}} \rfloor$, where $n$ is a number of features in the classification problem. This yields that in problem with 10 different variables, only 3 randomly chosen will be considered for splitting the node. Randomness parameter, so the number of chosen variables has a meaningful impact on the model, because except controlling the amount of randomness within each tree, it controls also the amount of correlation between different trees in the forest. As randomness parameter decreases, trees become more decorrelated [Decision Forests: A Unified Framework for Classification, Regression, Density Estimation, Manifold Learning and Semi-Supervised Learning].
\subsection {Training, testing and prediction}
Training of all trees is done independently and testing consists in the fact that each test point $\textbf{x}$ is pushed through every tree included in forest until it ends in corresponding leaves. As a last step is taking all predictions from every unit decision tree and combining them into prediction of single random forest. Combination of different tree predictions can be done in different ways. In random forests for classification problems, forests generate probabilistic output. It means that they return not just a single class point prediction, but whole class distribution, so combination of tree predictions can be described as below:
\begin{equation}
p(c|\textbf{x}) =  \frac{1}{T} \displaystyle\sum_{t}^{T} p_{t}(c|\textbf{x})
\end{equation}
where in a random forest with the number of decision trees equals to $T$ each tth tree obtains the posterior distribution $ p_{t}(c|\textbf{x})$. The class label here is symbolized by $c$ such that $c \in \textbf{C}$ with $ \textbf{C} = \{ c_{k}\} $
\subsection {Out of Bag sample}
Another important idea which is used in applications of random forest is validating model using Out of Bag sample. When performing a bootstrap for getting a sample of data for training, remaining part of data is our Out of Bag (OOB) sample. After training random forest model, OOB sample will be used as not known data for prediction. Leftover sample will be passed through every possible decision tree in the model that not include this data in the bootstrap training sample. Out of Bag is commonly used to monitor error of predictions. Breiman’s work on error estimates from 1996, shows empirically that Out of Bag estimate is as accurate as a test set of the same size as the training set [Leo Breiman, OUT-OF-BAG ESTIMATION, 1996].


\section{Mathematical explanation}

Let $D = {(x_{1},y_{1}), (x_{2}, y_{2}), ... , (x_{N}, y_{N})}$ be the set, we want to train our model on and $T_{D, \theta}$ decision tree produced by using the set $D$ and parameters $\theta$. We assume that $D$ is countable which normally is the case especially for Y values although replacing sums with integrals can extent the analysis and provide results for uncountable sets as well (Kohavi, 1996). As mentioned earlier, Random Forest classifier selects a bootstrapped subset of observations denoted as $\boldsymbol{D'}$ and grow the decision tree with only a subset of regressors. Repeating this tree growing process $B$ times gives a Random Forest Estimator. Assume $x^*$ is the value that we want to predict its class, there are two rules that can be used to get the prediction; majority voting and soft voting (Louppe, 2014; Zhou,2012).

In majority voting, after getting every trees prediction denoted as $\hat{T}(x^*)$ final prediction is the class that gets most votes from trees;
\begin{equation}
\boldsymbol{RF}_{D, \theta_{1}, \theta_{2}, ..., \theta_{B}} (x^*) =
	\underset{c \in Y}{argmax} \sum_{b = 1}^{B}{1(\hat{T}_{b}(x^*) = c)}
\end{equation}


In soft voting, probability estimates of a tree denoted as $\hat{p}_{D, \theta_{b}} (Y = c | X = x^*)$ is estimated and after all are averaged and most likely class is predicted;
\begin{equation}
\boldsymbol{RF}_{D, \theta_{1}, \theta_{2}, ..., \theta_{B}} (x^*) =
	\underset{c \in Y}{argmax} \dfrac{1}{B}\sum_{b = 1}^{B}{\hat{p}_{D, \theta_{b}} (Y = c | X = x^*)}
\end{equation}


As mentioned in Breiman(1994), aformentioned two voting procedures provides similar results, yet using soft voting can provide smoother class probability estimates and be exploited in a deeper analysis setting such as certainty estimates investigation (Louppe, 2014). 

\subsection{Properties}

The generalization error which also called test error or the expected prediction error of $T_{D,\theta}$ is;
\begin{equation}
\boldsymbol{Err}(T_{D,\theta}) = \mathbb{E}_{X,Y}\{L(Y, T_{D,\theta}(X)) \}
\end{equation}
where L is the loss function measuring the difference between its two arguments. Since we focus on classification setting, the zero-one loss function is our interest, however, to get a better understanding, widely used in regression type predictions, the squared loss function will be examined first. The bias-variance decomposition of both functions are similar and follow the same dynamics(Domingos, 2000). The squared loss function can be defined as
\begin{equation}
L(Y, T_{D, \theta}(x)) = (Y - T_{D, \theta}(x))^2
\end{equation}
while the zero-one loss function is
\begin{equation}
L(Y, T_{D,\theta}(X)) = 1 (Y \neq T_{D, \theta}(X))
\end{equation}
The expected prediction error for both functions 
\begin{align}
\boldsymbol{Err}(T_{D,\theta}) & = \mathbb{E}_{X,Y}\{ (Y - T_{D, \theta}(x))^2 \} \\
\boldsymbol{Err}(T_{D,\theta}) & = \mathbb{E}_{X,Y}\{ 1(Y \neq T_{D, \theta}(X)) \}
= P(Y \neq T_{D, \theta}(X))
\end{align}
where the last expression in the second equation is the probability of misclassification of the tree.

Given the probability distribution of P(X,Y), there exists a model $\phi_{\beta}$ that minimizes the expected prediction error and can be derived analytically independent or learning set $D$ (Louppe, 2014). Conditioning on X gives;
\begin{equation}
\mathbb{E}_{X,Y} \{L(Y, \phi_{\beta}(X))\} = \mathbb{E}_{X}\{\mathbb{E}_{Y|X}\{L(Y, \phi_{\beta}(X)) \} \}
\end{equation}
Minimizing the term with conditional expectation with respect to Y;
\begin{equation}
\phi_{\beta} = \underset{c \in Y}{argmin} \mathbb{E}_{Y|X=x}\{L(Y,c)\}
\end{equation}
$\phi_{\beta}$ is defined as Bayes model and $\boldsymbol{Err}(\phi_{\beta})$ is the residual error, the minimum obtainable error with any model, which is considered as the irreducible error due to random deviations in the data(Louppe, 2014). We will exploit the irreducible concept when examining the dynamics of random forests and decision trees. In that sense, with couple of manipulations, Bayes Model for squared loss is
\begin{align}
\phi_{\beta} & = \underset{c \,\in\, Y}{argmin} \; \mathbb{E}_{Y|X=x}\{(Y-c)^2 \} \notag\\
			 & = \mathbb{E}_{Y|X=x}\{Y\}
\end{align}
For squared loss function, bayes model predicts the average value of Y at X=x. In zero-one loss function case Bayes Model is
\begin{align}
\phi_{\beta} & = \underset{c \,\in\, Y}{argmin} \; \mathbb{E}_{Y|X=x}\{L(Y,c) \} \notag\\
			 & = \underset{c \,\in\, Y}{argmax} \; P(Y = c \,| X = x)
\end{align}
The most likely class in the set Y is chosen by Bayes Model when using the zero-one loss function. Aformentioned residual error can be computed for both functions;
\begin{align}
\boldsymbol{Err}(\phi_{\beta}) & = \mathbb{E}_{Y|X=x}\{(Y-\phi_{\beta}(x))^2 \}\\
\boldsymbol{Err}(\phi_{\beta}) & = P(Y \neq \phi_{\beta}(x) )
\end{align}

With using the squared loss function, $\boldsymbol{Err}(T_{D,\theta}(x))$ can be written as
\begin{align}
\boldsymbol{Err}(T_{D, \theta}(x)) & = \mathbb{E}_{Y|X=x}\{(Y-T_{D,\theta}(x))^2\} \notag \\
							   	  & = \boldsymbol{Err}(\phi_{\beta}(x)) + (\phi_{\beta}(x)-T_{D,\theta}(x))^2
\end{align}
Intermediate steps are included in the Appendix. As mentioned above, first term in the equation corresponds the irreducible error and the second term is due to the prediction differences between Bayes Model and our decision tree estimation. The expected prediction error increases with an increase in that difference. Since the result does not depend on the Y-values, it can be also expressed without conditional expectation. Decision trees in random forest classifier uses a bootstrapped dataset and $D$ is a random variable, thus, if we further examine the second term with taking expectation over $D$, it decomposes as
\begin{align}
& \mathbb{E}_{D}\{(\phi_{\beta}(x) - T_{D,\theta}(x))^2 \} \notag \\
&= (\phi_{\beta}(x) - \mathbb{E}_{D}\{T_{D,\theta}(x))^2 + \mathbb{E}_{D}\{(\mathbb{E}_{D}\{T_{D,\theta}(x)\} - T_{D,\theta}(x))^2\}
\end{align}
With intermediate steps being in Appendix, the first term in $(23)$ shows how the expected prediction of our decision tree differs from Bayes Model also called squared bias and the latter term is the variance of our estimator. Therefore, we can define $\boldsymbol{Err}(T_{D,\theta})$ as follows
\begin{equation}
\boldsymbol{Err}(T_{D,\theta}) = noise(x) + bias^2(x) + var(x)
\end{equation}
where
\begin{align}
& noise(x) = \boldsymbol{Err}(\phi_{\beta}) \notag \\
& bias^2(x) = (\phi_{\beta}(x) - \mathbb{E}_{D}\{T_{D,\theta}(x)\})^2 \notag \\
& var(x) = \mathbb{E}_{D}\{(\mathbb{E}_{D}(T_{D,\theta}(x)) - T_{D,\theta}(x))^2\} \notag
\end{align}
The same decomposition can be conducted for the zero-one loss function and as mentioned in (Louppe,2014), (Domingos,2000), (James,2003), (Friedman,1997) both zero-one and squared loss functions can be decomposed in similar fashion. However, since the distribution of $D$ is unknown, bias-variance decompostion cannot be solved explicitly as done for squared loss(Louppe, 2014). (Kohavi,1996) introduces another decomposition for zero-one loss function (included in the Appendix), but, still it remains to be unexplanatory compared to squared loss, thus, we explain the dynamics with using squared loss although the main focus of the paper remains to be on classification setting.

In regression setting, random forest classifier shares the same idea with classification prediction with soft voting. Random forest classifier for regression can be written as
\begin{equation}
\boldsymbol{RF}_{D, \theta_{1},\theta_{2},..., \theta_{B}}(x) = \dfrac{1}{B}\sum_{b = 1}^{B}(T_{D,\theta_{b}}(x))
\end{equation}
When we take the average prediction in this case equals to expectation in terms of training set, we get
\begin{align}
\mathbb{E}_{D, \theta_{1}, \theta_{2},..., \theta_{B}}\{\boldsymbol{RF}_{D, \theta_{1},\theta_{2},..., \theta_{B}}(x) \} & = 
	\mathbb{E}_{D, \theta_{1}, \theta_{2},..., \theta_{B}}\{\dfrac{1}{B}\sum_{b = 1}^{B}(T_{D,\theta_{b}}(x))\} \notag \\
	& = \dfrac{1}{M}\sum_{b=1}^{B}\mathbb{E}_{D,\theta_{b}}\{T_{D, \theta_{b}}(x)\}\notag \\
	& = \mu_{D,\theta}(x)
\end{align}
where $\mu_{D,\theta}(x)$ is the average prediction of all ensembled trees as a random forest since $\theta$'s are random, independent
and have the same distribution(Louppe, 2014). When we extend this finding bias of a random forest we can state that 
\begin{equation}
bias^2(x) = (\phi_{\beta}(x) - \mu_{D,\theta}(x))^2
\end{equation}
meaning that squared bias cannot be decreased and will be same for any randomized models. Although random forest is inadequate to propose any structure to decrease the prediction error so far regarding $noise(x)$ and $bias^2$, it displays phenomenal performance in reducing the last remaining part of the prediction error. Thus, we can continue our exploration with variance of random forest and need to define the correlation coefficient $\rho(x)$. For any two trees $T_{D,\theta'}$ and $T_{D,\theta''}$ trained with the same training data and different growing parameters $\theta'$ and $\theta''$, we can define the correlation coefficient as follows
\begin{align}
\rho(x) & = \dfrac{\mathbb{E}_{D,\theta',\theta''}\{T_{D,\theta'}(x) T_{D,\theta''}(x)\} - \mu_{D,\theta}^2(x)}{\sigma_{D,\theta}^2(x)}
\end{align}
We utilize the definition of the Pearson's correlation coefficient and the property of $\theta'$ and $\theta''$ following the same distribution in the intermediate steps which are included in the Appendix. $\sigma_{D, \theta}(x)$ is the variance of a decision tree and generally $\rho(x)$ represents the effect of randomization in the learning algorithm. In our case, it is close to 1 when predictions of two decision trees are highly correlated and implies that randomization does not have a significant effect. On the other hand, if it is close to 0, the prediction of two trees are perfectly random and non-correlated.

We can decompose variance of a random forest as follows;
\begin{align}
\mathbb{V}_{D, \theta_{1}, \theta_{2},..., \theta_{B}}\{\boldsymbol{RF}_{D, \theta_{1},\theta_{2},..., \theta_{B}}(x) \}  = \rho(x)\sigma^2_{D,\theta}(x) + \dfrac{1-\rho(x)}{B}\sigma^2_{D,\theta}(x)
\end{align}
Increasing the number of ensembled trees $B$, will lower the latter expression in the equation and in the extreme case where $B \rightarrow \infty $, the variance of a random forest equals to $\rho(x)\sigma^2_{D,\theta}(x)$ and due to randomization in the algorithm $\rho(x) < 1$, thus, variance of a random forest is less than variance of a decision tree. This inference implies that the expected prediction error of a random forest is less than the expected prediction error of a decision tree. If the decision trees in the random forest are independent and consequently $\rho(x) \rightarrow 0$, the variance is reduced to $\dfrac{\sigma^2_{D,\theta}(x)}{B}$ which can be decreased with increasing $B$ as mentioned.



\section{Interpretation}
In many cases, the main purpose of using a random forest model is its usefulness in performing predictions of a dependent variable based on a set of explanatory variables. In addition, very often, to understand the processes under study we need to understand which explanatory variables are the most important and useful to make mentioned predictions. Thanks to random forest we are often capable to not only build an accurate model with reliable predictions, but also to provide variable importance measures which are very important in the process of interpreting the model and its prediction results. This section of the paper will be fully based on work “Understanding variable importances in forests of randomized trees” presented by G. Louppe, L. Wehenkel, A. Sutera and P. Geurts and "A Random Forest Guided Tour" presented by G. Biau and E. Scornet.

\subsection{Variable importance}
In order to rank the importance of the explanatory variables in classification problems using random forest we use two possible measures. Both the first and second measure have been proposed by Breiman (2001, 2002). First of them is known as Mean Decrease Impurity (MDI). MDI measure is based on the total decrease in node impurity from splitting on the given explanatory variable and moreover MDI is averaged over all trees. The second measure is known as Mean Decrease Accuracy (MDA). This measure assumes that if the explanatory variable is not relevant to the study of the problem, then rearranging its values should not demolish prediction accuracy. Mean Decrease Impurity proposed by Breiman (2001, 2002) is given by: 
\begin{equation}
\widehat{MDI}( X^{(j)} ) = \frac{1}{N_{T}}\displaystyle\sum_{T}  \displaystyle\sum_{t \in T: v(s_{t}) =  X^{(j)}  } p(t)\Delta i(s_{t}, t)
\end{equation}
where $ v(st) $ is the variable used in split $s_{t}$ and $ p(t) $ is the proportion $\frac{N_{t}}{N}$ of samples reaching $t$.
By definition $ MDI( X^{(j)} ) $ evaluates importance of a variable $ X^{(j)} $ for predicting $Y$ by adding up the weighted impurity decreases $p(t)i(s_{t}, t)$ for all nodes $t$ where $ X^{(j)}$ is used. Mean in the name of MDI comes from averaging over all $N_{T}$ trees in the random forest. Above definition can be used for any impurity measure $i(t)$ for example: the Gini index, the Shannon entropy. The second mentioned measure MDA uses out-of-bag error estimate. In this case to measure the importance of the $X^{(j)}$ variable, we have to permute its values in the out-of-bag example and use these all observations in the tree. $ MDA( X^{(j)})$ is counted thanks to averaging the differences in out-of-bag error estimation after and before the permutation in all trees. Because of used permutations, this measure is sometimes called also as the permutation importance. In order to represent MDA measure mathematically, consider an explanatory $j-th$ variable $X^{(j)}$ and stand $OOB_{k,n}$ as a symbol for the out-of-bag data set of the $k-th$ tree and  $OOB_{k,n}^{j}$ as the same data set, but with randomly permuted values of $X^{(j)}$. Denote also by $ m_{n}(\cdot ;\Theta_{k}) $  the $k-th$ tree estimate. Then definition of MDA is given by:

\begin{equation}
\widehat{MDA}( X^{(j)} ) = \frac{1}{N_{T}}\displaystyle\sum_{T} \Big[C_{n}\big[m_{n}(\cdot ;\Theta_{k}), OOB_{k,n}^{j}\big] - C_{n}\big[m_{n}(\cdot ;\Theta_{k}), OOB_{k,n}\big]   \Big],
\end{equation}
where $R_{n}$ is defined for $ OOB =  OOB_{k,n}^{j}$ or $OOB =  OOB_{k,n}$ by:
\begin{equation}
C_{n}\big[m_{n}(\cdot ;\Theta_{k}), OOB\big] = \frac{1}{|OOB|} \displaystyle\sum_{ i :( \pmb{X_{i}}, Y_{i} )  \in OOB} (Y_{i} - m_{n}( \pmb{X_{i}} ;\Theta_{k})).
\end{equation}

Above formulas are true for regression and classification purposes. It is also visible that population version for $\widehat{MDA}( X^{(j)} )$ is as given below:
\begin{equation}
MDA( X^{(j)} ) =  \mathbb{E}[Y - m_{n}( \pmb{X^{'}_{j}};\Theta_{k})]^{2} -  \mathbb{E}[Y - m_{n}( \pmb{X};\Theta_{k})]^{2},
\end{equation}

where $ \pmb{X^{'}_{j}} = (X^{(1)},...,X^{'(j)},...,X^{(p)}) $ and $X^{'(j)}$ is here independent copy of  $X^{(j)}$. 











\chapter{Application and Comparison}
This chapter covers the application of random forest regression and the evaluation of its performance.

In \autoref{sec:simulation}, we apply the random forest on simulated data, 
and show how its performance develops over increasing sample sizes.

Then in \autoref{sec:real_data}, we apply the random forest on the real Titanic data set \cite{titanicData}
and evaluate its performance.

In \autoref{sec:gradient_boosting} and \autoref{sec:gradient_boosting}, we apply AdaBoost and Gradient
Boosting respectively on the Titanic data set and  compare their performance with that of the random forest.

\section{Application of random forest}
\subsection{Simulated data}
\label{sec:simulation}
In the simulation, we use a linear and a non-linear data generating process (DGP) for random forest regression.
The linear DGP generates the data tuples \( (y, x_{1}, x_{2}, x_{3}) \) as follows:

\begin{equation}\label{eq:linear_dgp}
    y = \beta_{0} + \beta_{1} x_{1} + \beta_{2} x_{2} + \beta_{3} x_{3} + \epsilon,
\end{equation}

whereas \( (\beta_{0}, \beta_{1}, \beta_{2}, \beta_{3}) = (0.3, 5, 10, 15) \),
\( x_{1}, x_{2}, x_{3} \sim \mathcal{N}(0,\,3) \), and \( \epsilon \sim \mathcal{N}(0,\,1) \).

The performance of the Random Forest over an increasing sample is illustrated
below in \autoref{fig:forest_vs_ols_nonLinearDGP} and \autoref{fig:forest_vs_ols_nonLinearDGP}.
For each sample drawn from the linear DGP, a set of parameters
were optimized via cross validation. Then, the residual sum of squares (RSS) gets calculated
based on the holdout set of 100 instances.

\begin{figure}[H]
    \captionsetup{format=plain}
    \makebox[\textwidth]{\includegraphics[width=120mm]{forest_vs_ols_linearDGP.png}}
    \caption
        {This plot illustrates the RSS for different training sample sizes for Random Forest and OLS.
        These samples where drawn from a linear DGP in accordance to \autoref{eq:linear_dgp}.
        The holdout set for calculating the RSS were drawn again for each training sample from the same DGP.
        It always contained 100 observations. In case of the Random Forest, for each sample the parameters
        got optimized again via cross validation.
        }
    \label{fig:forest_vs_ols_linearDGP}
\end{figure}

As one can see in \autoref{fig:forest_vs_ols_linearDGP} above, the RSS of the Random Forest converges for the linear DGP
to that of the OLS for increasing sample sizes. 

The non-linear DGP generates the data tuples \( (y, x_{1}, x_{2}) \) as follows:

\begin{equation}\label{eq:non_linear_dgp}
    y = \beta_{0} + \beta_{1} I(x_{1} >= 0, x_{2} >= 0) + \beta_{2} I(x_{1} >= 0, x_{2} < 0) + \beta_{3} I(x_{1} < 0) + \epsilon,
\end{equation}

whereas \( (\beta_{0}, \beta_{1}, \beta_{2}, \beta_{3}) \), \( x_{1}, x_{2} \) and \(  \epsilon \)
are the same in the previous DGP.

\begin{figure}[H]
    \captionsetup{format=plain}
    \makebox[\textwidth]{\includegraphics[width=120mm]{forest_vs_ols_nonLinearDGP.png}}
    \caption
        {This plot illustrates the RSS for different training sample sizes for Random Forest and OLS.
        These samples where drawn from a non-linear DGP in accordance to \autoref{eq:non_linear_dgp}.
        The holdout set for calculating the RSS were drawn again for each training sample from the same DGP.
        It always contained 100 observations. In case of the Random Forest, for each sample the parameters
        got optimized again via cross validation.
        }
    \label{fig:forest_vs_ols_nonLinearDGP}
\end{figure}

As one can see above in \autoref{fig:forest_vs_ols_nonLinearDGP}, the Random Forest performs strictly better
than the OLS for any sample size. Due to this DGP resembling a stratification similar to
that of a Descision Tree, the RSS of the Random Forest converges relatively quickly while that of the OLS
remains unstable and high. 


\subsection{Real data example}
\label{sec:real_data}
As previously mentioned, we applied the Random Forest on the Titanic data set \cite{titanicData} in order
to determine the survival of the passengers based on reported attributes like name title or booked cabin. 
In order to use the data to its fullest extent, we conducted additional feature engineering. Without that, many
features remain unusable for our methods, because they contain missing values or values that are formatted as text.
For the implementation of the feature engineering and the classification, one can consult our code repository \cite{githubApplication}.
The Random Forest managed to achieve a total classification accuracy of 84,32\% on the holdout set.
The holdout set consists of 15\% of the total data.
According to the confusion matrix in \autoref{fig:confusion_matrix}, for passengers that died, 
the accuracy was slightly higher compared to those that survived. This is to be expected,
since deaths outnumber survivals considerably.

\begin{figure}[H]
    \captionsetup{format=plain}
    \makebox[\textwidth]{\includegraphics[width=200mm]{confusion_matrix.png}}
    \caption
        {This plot illustrates the accuracy of the Random Forest's prediction on the Titanic data set.
        }
    \label{fig:confusion_matrix}
\end{figure}



\section{AdaBoost Classifier}
\label{sec:adaboost}
Adaptive Booosting (Adaboost) was introduced by Freund and Schapire (1997) and we employ Adaboost as a comparison tecnique against Random Forest. In adaboost algorithm, instead of fully-grown trees as in random forest  we use trees with only one internal node and two leaves also called stumps. We consider those as weak-learners compared to trees since its depth and power is limited. Before generating stumps, we assign weights to observations in the sample, normally weight of a observation takes the value of $1/N$ as $N$ is the sample size. We generate a stump for each classifier in the data and compare them regarding their misclassification rate. After selecting the best classifier, with using its stump's misclassification rate we compute its stump's significance. With that significance we compute new weights for the sample. We repeat the algorith sequantially for the sample with new weights until a stopping criteria is achieved. Generally, using number of classifiers as iteriation number is a common practice (Elements of statistical learning). After generating multiple stumps, we can predict an observation's class. We get the decision of every stump and form groups accordingly. Every outcome class has a group of stumps which predict that class and every stump has its own significance. After summing significance values of stumps in each group, the prediction is the class with the total highest significance. Although stumps are weak-learners, we exploit the error of a weak-learner to generate another weak-learner and iterating this multiple times provides us a powerful algorithm.

\section{Gradient Boosting Classifier}
\label{sec:gradient_boosting}
\subsection{An introduction to the Gradient Boosting method}
Firstly, it is worth to remind what boosting is. Like bagging, boosting is an approach which can be applied to many machine learning methods for classification or regression. Bagging uses bootstrap to create multiple datasets for training the method. As a next stage bagging fits a separate decision tree to each training dataset, and then it combines all decision trees in order to create a single predictive model. Every decision tree is independent to others thanks to using bootstrap to create different training datasets. Boosting method works similarly, but in our case decision trees are grown sequentially. It means that each tree is built using information from previously built trees. Boosting method does not involve bootstrap sampling. In this method instead of bootstrap each decision tree is fit on a modified version of the original dataset.

In gradient boosting the idea is to take weak learning algorithm or hypothesis and make some corrections that will improve the power of this algorithm/hypothesis. In hypothesis boosting, you check every observation on which statistical learning method was trained on, then you leave the observations which were correctly classified. Then method creates new weak learner and test it only on the observations that were poorly classified. Next, the examples that were correctly classified are kept.
The idea described above was used in the AdaBoost (Adaptive Boosting) algorithm. In this algorithm, many weak learners are created thanks to many decision trees that only have a single split. Created instances in the training dataset are weighted in the way that larger weights are assigned to instances which were difficult to classify. To the most difficult training instances weaker learners are added sequentially.
Gradient boosting classifiers are the Adaptive Boosting method, but it is combined with weighted minimization. After weighted minimization, all classifiers and weighted inputs are again calculated. The aim of gradient boosting classifiers is to minimize the loss and it operates in the similar way, as gradient descent in a neural network.
\subsection{Real data example}

After a simple and short introduction it is time to apply Gradient Boosting Classifier on the Titanic data set. As in every application in the paper,
after conducting feature engineering, we try to determine the survival of the passengers using available explanatory variables. 
Gradient Boosting Classifier if it comes to a total classification accuracy is slightly worse than the Random Forest method. In this case achieved accuracy is equal
approximately 82,84\%. 
As for the Random Forest confusion matrix \autoref{fig:confusion_matrix_gradient_boosting} shows higher accuracy for passengers that died than passengers who survived and it is respectively 85\% and 80\%.

\begin{figure}[H]
\captionsetup{format=plain}
\makebox[\textwidth]{\includegraphics[width=200mm]{confusion_matrix_gradient_boosting.png}}
\caption
{This plot illustrates the accuracy of the Gradient Boosting Classifier's prediction on the Titanic data set.
}
\label{fig:confusion_matrix_gradient_boosting}
\end{figure}




\section{Conclusion and outlook}

We examined and applied one of the infamous machine learning algorithms, Random Forest. 
We started with explaining the Decision Trees and the room for improvement in it. 
As solutions of high-variance in Decision Trees, bagging, boosting and Random Forest are developed. 
Random Forest shows better predictions can be achieved with introducing randomness into the picture.
That randomness provides us with a decorrelated ensemble of trees and 
the increase in the number of trees yields lower error considering prediction purposes.
With using Random Forest, we are able to assess the importance of every variable and draw conclusions about data.
We explained the intuition of Random Forest in detail and included mathematical clarification. 
Finally, we applied Random Forest on both simulated and real data and compared with various methods. 
Considering results, Random Forest appears to be employed in the future as well, 
thus, a better understanding of the idea and the dynamics can give us a chance to improve. 
This paper aims to introduce the idea in detail and essentially a review and a showcase of the Random Forest algorithm. 

\printbibliography


\chapter*{Declaration}
I hereby certify that this material is my own work, that I used only those sources and resources referred to in the thesis, and that I have identified citations as such.
		
\vspace{0.3in}

\noindent Bonn, \today

\end{document}
